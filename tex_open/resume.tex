\chapter*{Résumé}
\addcontentsline{toc}{chapter}{Résumé}

%

%

\selectlanguage{french}
La présente thèse de doctorat explore l'interaction complexe entre les excitons et les phonons dans les nanostructures de nitrure de bore hexagonal (hBN) à l'aide de méthodes de calcul avancées. La thèse commence par une introduction au hBN, mettant en lumière ses propriétés uniques et son importance dans la physique de la matière condensée. 

Le chapitre 2 donne un aperçu complet du cadre théorique de pointe utilisé tout au long de la recherche, englobant la théorie de la fonctionnelle de la densité (DFT), la théorie des perturbations de la fonctionnelle de la densité (DFPT) pour les propriétés des phonons, et la théorie des perturbations à N-corps (MBPT) à travers l'approximation GW et l'équation de Bethe-Salpeter pour prendre en compte les effets collectifs électroniques et excitoniques.

Dans le chapitre 3, l'accent est mis sur le hBN massif soumis à une déformation uniaxiale, où le couplage exciton-phonon est étudié à l'aide d'une méthode de dérivation par différences finies. Cette approche reproduit approximativement les changements d'intensité de la luminescence observés lorsqu'une contrainte est appliquée au cristal. Les résultats de ce chapitre donnent des indications précieuses sur les propriétés électroniques, phononiques et excitoniques, ainsi que sur les interactions exciton-phonon dans les systèmes hBN massifs soumis à une déformation uniaxiale.

Le chapitre 4 se penche sur l'étude du hBN monocouche, en utilisant une méthode ab initio fondée sur une dérivation théorique rigoureuse des éléments de matrice du couplage exciton-phonon. En incorporant les pics directs et indirects dans les spectres de luminescence, cette méthode donne des intensités relatives détaillées, permettant une interprétation précise des mesures expérimentales publiées par différents groupes de recherche. Notamment, cette étude élimine la possibilité d'observer des répliques de phonons dans les spectres du hBN monocouche, apportant une nouvelle clarté à des interprétations auparavant ambiguës.

En outre, la thèse présente des résultats préliminaires pour le BN Bernal, un polytype de hBN avec un empilement de couches différent, présentant des gaps d'énergie directs et indirects très proches les uns des autres. Ce matériau intrigant a le potentiel d'afficher simultanément des pics directs et indirects dans les spectres de luminescence. Dans le cadre des recherches en cours qui incluent une étude numérique poussée, ce chapitre ouvre la voie à une compréhension plus approfondie des interactions exciton-phonon dans la phase Bernal du BN.

Dans l'ensemble, cette thèse de doctorat contribue de manière significative au domaine de la physique computationnelle de la matière condensée en clarifiant les phénomènes complexes de couplage exciton-phonon dans diverses nanostructures hBN. Les connaissances acquises grâce à cette étude peuvent faire progresser la compréhension et la conception de nouveaux dispositifs optoélectroniques basés sur des matériaux hBN.

\vspace{0.5cm}
Mots-clés: luminescence, phonons, excitons, couplage exciton-phonon