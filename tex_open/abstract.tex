\chapter*{Abstract}
\addcontentsline{toc}{chapter}{Abstract}
\selectlanguage{english}
%In the introduction, I present what is hexagonal Boron Nitride (hBN). In chapter 2, I present the state-of-the-art theory that I used for my work. The starting point is Density Functional Theory, then I compute phonon properties with Density Functional Perturbation Theory. Besides, the electronic and excitonic effects are included with Many-Body Perturbation Theory calculations through the GW approximation and the Bethe-Salpeter equation. I used two different methods to compute the exciton-phonon coupling, which were applied to two different systems. The first system is strained bulk hBN. I used a finite-difference derivative method in order to reproduce the change in luminescence intensity when strain is applied to the crystal.  I present it in chapter 3. In chapter 4, I used an ab initio method based on a rigorous theoretical derivation of the exciton-phonon coupling matrix elements. This method allows to include both direct and indirect peaks in the luminescence spectra with their relative intensities, this is why we applied it to a direct-gap material : monolayer hBN. We tried to explain the features appearing in the experimental measurements published recently by different groups who had various interpretation. We ruled out the possibility of seeing the phonon replicas in the spectra for this material.  We also present the preliminary results on a polytype of hBN, the so-called Bernal BN, which has the particularity of having direct and indirect gaps very close in energy and could possibly exhibit both direct and indirect peaks in luminescence spectra.

The present PhD thesis explores the intricate interplay between excitons and phonons in hexagonal Boron Nitride (hBN) nanostructures through advanced computational methods. The thesis commences with an introduction to hBN, shedding light on its unique properties and relevance in condensed matter physics. 

Chapter 2 provides a comprehensive overview of the state-of-the-art theoretical framework employed throughout the research, encompassing Density Functional Theory (DFT), Density Functional Perturbation Theory (DFPT) for phonon properties, and Many-Body Perturbation Theory (MBPT) through the GW approximation and the Bethe-Salpeter equation to account for collective electronic and excitonic effects.

In Chapter 3, the focus lies on uniaxially strained bulk hBN, where the exciton-phonon coupling is studied using a finite-difference derivative method. This approach approximately reproduces the changes in luminescence intensity observed when strain is applied to the crystal. The outcomes of this chapter offer valuable insights into the electronic, phononic and excitonic properties, as well as exciton-phonon interactions in bulk hBN systems under uniaxial strain.

Chapter 4 delves into the investigation of monolayer hBN, employing an ab initio method grounded on a rigorous theoretical derivation of the exciton-phonon coupling matrix elements. By incorporating both direct and indirect peaks in the luminescence spectra, this method yields detailed relative intensities, enabling an accurate interpretation of experimental measurements published by different research groups. Notably, this study eliminates the possibility of observing phonon replicas in the spectra of monolayer hBN, providing new clarity to previously ambiguous interpretations.

Furthermore, the thesis offers preliminary results for Bernal BN, a polytype of hBN with a different layer stacking, featuring closely situated direct and indirect energy gaps. This intriguing material holds potential for displaying simultaneously both direct and indirect peaks in luminescence spectra. As part of ongoing research that includes deep numerical studies, this chapter paves the way for a deeper understanding of exciton-phonon interactions in Bernal BN structures.

Overall, this PhD thesis contributes significantly to the field of computational condensed matter physics by unraveling the complex exciton-phonon coupling phenomena in various hBN nanostructures. The insights gained from this study have the potential to advance the understanding and design of novel optoelectronic devices based on hBN materials.

\vspace{0.5cm}
Keywords: luminescence, excitons, phonons, exciton-phonon coupling

