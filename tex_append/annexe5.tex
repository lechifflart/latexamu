\section{Computational details for Chapter 3} \label{app:comp_details_Chapt3}
In this section we present our computational workflow for the calculation of the exciton-phonon coupling and all the computational details needed to reproduce the results. In Fig. \textbf{Fig} \textcolor{red}{huuustre} we report a sketch of the workflow.
We start from the atomic structure, which was optimized in DFT for the monolayer and taken from Ref. \cite{sponza2018direct} for bulk hBN. The relevant lattice parameters are shown in Table~\ref{tab:parms}. We performed all DFT calculations using the \textsc{Quantum ESPRESSO} code\cite{giannozzi2009quantum}, with norm-conserving pseudopotentials\cite{van2018pseudodojo} in the Local Density Approximation (LDA).\cite{PhysRevB.43.1993} In the monolayer case, a supercell was used with a length of 21~\r{A} along the $z$-direction -- in such a way to avoid spurious interactions between periodic images -- along with the 2D cutoff for the Coulomb interaction implemented in Quantum Espresso.\cite{sohier2017density} Energy cutoff and other parameters that enter the DFT calculations are reported in Tab. \ref{tab:parms}. Phonons and electron-phonon couplings were calculated using DFPT starting from the DFT results. The transferred momenta grid for the phonons is reported in Tab. \ref{tab:parms}. The correct long-range behavior of the electron-phonon coupling in 2D was obtained by applying a cutoff of the Coulomb interaction in the $z$-direction.\cite{sohier2016two} The electron-phonon matrix elements were calculated on the same $\qq$-grid as the phonons and the excitons, for all the electronic bands entering in the Bethe-Salpeter equation kernel.

Using the Kohn-Sham band structure as a starting point, we employed MBPT as implemented in the \yambo~ code\cite{Sangalli_2019} to calculate quasiparticle band structures within the G$_0$W$_0$ approximation\cite{aryasetiawan1998gw} and the excitonic optical response functions using the Bethe-Salpeter Equation.\cite{strinati1988application}
All the many-body operators that enter in these calculations are expanded in a Kohn-Sham basis set. Therefore, in order to have converged results, we diagonalized the Kohn-Sham Hamiltonian for a large number of bands that were then used to build the electronic Green's function $G$ and the dielectric matrix $\epsilon$. In table \ref{tab:parms} we also report the cutoff used in the construction of the dielectric matrix. Finally, in order to speed up convergence with respect to the empty states, we used a terminator for both $\epsilon$ and $G$.\cite{bruneval2008accurate}
The BSE was constructed using a static kernel derived from the GW self-energy within the Tamm-Dancoff approximation.\cite{strinati1988application}  We include only the lowest conduction and the highest valence bands in such a way to get converged absorption and emission spectra. The BSE was solved for a grid of transferred momenta $\QQ$ identical to the phonons grid. 

\begin{table}
    \begin{center}
        \begin{tabular}{ | c | c | c |}
        \hline
             Parameter/System &  m-hBN & hBN (and bBN) \\ \hline
        \hline
        $\qq$/$\kk$-grid & $36 \times 36 \times 1$ & $18 \times 18 \times 6$  \\ \hline
        $a$ & 2.48 \r{A} & 2.50 \r{A}   \\ \hline
        $c$ & - & 3.25 \r{A}   \\ \hline
            GW/ $\varepsilon(\omega)$ bands   & 200 & 210 \\ \hline
            $\varepsilon(\omega)$ cutoff  & 10~Ha & 8~Ha \\ \hline
        BSE bands & 3-6 & 5-12 \\ \hline
        Excitons ($\beta \rightarrow \lambda$) & $8\rightarrow 2$ & $12\rightarrow 4$ \\ \hline
        Double grid & $108 \times 108 \times 1 $ & $54 \times 54 \times 18 $ \\ \hline
        \end{tabular}
        \caption{List of the relevant computational parameters entering the calculation of excitons, phonons and their coupling ($a$ is the planar lattice parameter and $c$ the interlayer distance).\label{tab:parms}}
    \end{center}
    \end{table}


Luminescence spectra were calculated using Eq.~\eqref{eq:I_PL}. We first built the exciton-phonon matrix elements using the results obtained from BSE and DFPT, as indicated in the scheme in Fig.~\textcolor{red}{put the scheme of the workflow}. We selected a number of ``initial'' excitons at finite $\QQ$ (indices $\beta$ in Eq.~\eqref{eq:I_PL}), that scatter with phonons (indices $\mu,\qq$) and end up in the ``final'' excitonic states at $\QQ=0$ (indices $\lambda$). All phonon modes and transferred momenta were included in these calculations. Note that both the the electronic and transferred momenta ($k$-grid and $q$-grid, respectively), were computed on the irreducible parts of the respective Brillouin zones (BZs).
The exciton-phonon coupling matrix elements $\mathcal{G}$ from Eq.~\eqref{eq:Gkkp} were then expanded in the full BZs by symmetry transformations applied to the electron-phonon matrix elements $g$ and the excitonic coefficients $A$. In this way, we are able to strongly speed up exciton-phonon calculations, which would otherwise require the switching off of all crystal symmetries at the DFPT and MBPT levels.

Then, we interpolated both exciton dispersions -- using a smooth Fourier interpolation\cite{pickett1988smooth} -- and phonon energies -- using the force constants method implemented in \textsc{Quantum ESPRESSO} -- on a finer grid to speed up convergence of the luminescence spectra with respect to the transferred momenta grid. Then these finer grids are used to average out the denominators appearing in the luminescence formula as:
\begin{equation}
\frac{1}{W^{\pm}_{\lambda, \beta,  \qq, \mu }} =\frac{1}{N_{\tilde q}} \sum_{\tilde q \in \qq }\frac{1/2 \pm 1/2 + n_{\tilde q,\mu} }{ \omega - (E_{\tilde q,\beta} \mp \omega_{\tilde q \mu}) + i\eta } e^{-\frac{E_{\tilde{q},\beta}-\tilde{E}_{min}}{kT_{exc}}}
\end{equation}
where $\tilde{E}_{min}$ is the minimum exciton energy on the double-grid, than will likely be different from the one on the coarse grid. Using the average denominators defined above we can write the luminescence as : \textcolor{green}{check if this is correct}
\begin{multline}
	{I}^{PL}(\omega)= \Im \sum_\lambda \frac{| T_\lambda  |^2}{\pi^2 \hbar c^3} \left\{ \omega^3 n_r(\omega) \frac{1 - R_\lambda}{\omega - E_\lambda + i\eta} e^{-\frac{E_{\lambda}-E_{min}}{kT_{exc}}} \right.\\
    \left. + \sum_{\mu\beta\qq} \omega(\omega \mp 2 \omega_{\qq \mu})^2 n_r(\omega) \left|\mathcal{D}^{\pm}_{\beta\lambda,\qq \mu} \right| \frac{1}{W^{\pm}_{\lambda, \beta,  \qq, \mu }} \right\},
\end{multline}
implemented as a sum over the fine-grid $\tilde q$-points in the neighborhood of each coarse-grid $q$-point.

