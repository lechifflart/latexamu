\section{Derivation of equations of motion for field operators}
\label{app:EOM}
Here we derive the equations of motion for the field operators in Heisenberg picture, based on \cite{martin2016interacting, stefanucci2013nonequilibrium, strinati1988application, aryasetiawan1998gw}. In the main text however, the interaction picture is used. The following derivation is left unchanged if one considers the unperturbed Hamiltonian to be the time-independent $\hat{H}$ and any other time-dependent external perturbation, which is exactly what was done in the main text.
We shall make explicit the time dependence of every term appearing in the Green's function in Eq. \eqref{eq:GF}. We start with the time evolution of the field operators. We recall some useful properties of the field operators for fermions in the Schrödinger picture :
\begin{align}
\begin{split}
	\{ \hpsi(x),\hpsidag(x')\} &= \delta (x-x') \\
	\{ \hpsi(x),\hpsi(x') \} &= \{ \hpsidag(x),\hpsidag(x') \} = 0 \\
	n(x) &= \hpsidag(x)\hpsi(x)
\end{split}	
\end{align}
 The total Hamiltonian enters the Heisenberg equation of motion for an operator $\hat{O}$:
\begin{equation}
	i\frac{d}{d t}\hat{O}_H(t) = \hat{U}^\dagger_S(t) \left[ \hat{O}(t),\hat{H} \right] \hat{U}_S(t) + \hat{U}^\dagger_S(t) (i \frac{d}{dt} \hat{O}_S(t)) \hat{U}_S(t)
\end{equation}
where the subscript $H$ and $S$ denote respectively the Heisenberg and Schrödinger pictures, and the transformation from the latter to the former is given by :
\begin{align}
\begin{split}
	\hpsi_H(x,t) &= \hat{U}^\dagger_S(t) \hpsi_S(x) \hat{U}_S(t) \\
	\hpsidag_H(x,t) &= \hat{U}^\dagger_S(t) \hpsidag_S(x) \hat{U}_S(t)
\end{split}
\end{align}
and $\hat{U}_S(t) = \exp(-i\hat{H}t) $ is the time evolution operator. In the following we drop the subscript $H$ for the field operators, as their time dependence will be explicit. The Heisenberg equation of motion for the field operator is then :
\begin{equation}
	i\frac{d}{dt} \hpsi(x,t) = \hat{U}^\dagger_S(t) \left[ \hpsi(x),\hat{H} \right] \hat{U}_S(t)
\end{equation}
and similarly for $\hpsidag$. To compute the commutator, we split the two terms of the Hamiltonian and we use the identity 
\begin{equation}
	\left[ \hpsi(x),\hat{A}\hat{B} \right] = \{ \hpsi(x),\hat{A} \}\hat{B} -  \hat{A}\{ \hpsi(x), \hat{B} \}
\end{equation}
where we take
\begin{align} 
\begin{split}
	\hat{A} &= \hpsidag(x_1) \\
	\hat{B} &= h(x_1) \hpsi(x_1)
\end{split}
\end{align}
Since $\{\hpsi(x),\hat{B}\} = 0$, then 
\begin{equation}
	\left[ \hpsi(x),\hat{H}_0\right] = h(x) \hpsi(x)
\end{equation}
Now we notice that the second term in the commutator contains 
\begin{align}
\begin{split}
	\left[ \hpsi(x), \hpsidag(x_1)\hpsidag(x_2)\hpsi(x_2)\hpsi(x_1) \right] &= \left[ \hpsi(x),\hpsidag(x_1)\hpsidag(x_2) \right] \hpsi(x_2)\hpsi(x_1) \\
	&= \left( \hpsidag(x_1) \delta(x_1-x_2) + \hpsidag(x_2)\delta(x-x_1) \right) \hpsi(x_2)\hpsi(x_1).
\end{split}
\end{align}
Therefore, 
\begin{align}
\begin{split}
	\left[ \hpsi(x),\hat{H}_{int} \right] &= \frac{1}{2} \int dx_1 \hpsidag(x_1)\hpsi(x)\hpsi(x_1) v(x,x_1) + \frac{1}{2} \int dx_2 \hpsidag(x_2)\hpsi(x_2)\hpsi(x) v(x,x_2) \\
	&= \int dx_2 v(x,x_2) \hpsidag(x_1) \hpsi(x_2) \hpsi(x)
\end{split}
\end{align}
where in the second line we used the symmetry property of the Coulomb interaction $v(x,x') = v(x',x)$.
Finally, with the compact notation $1 \equiv (\rr_1, \sigma, t_1)$, we get the equations of motion for the field operators :
\begin{align}
\begin{split}
	\frac{\partial}{\partial t_1} \hpsi(1) &= -i \left[ h(1) + \int d3 v(1,3) \hpsidag(3)\hpsi(3) \right] \hpsi(1) \\
	\frac{\partial}{\partial t_2} \hpsidag(2) &= i \left[ h(2)\hpsidag(2) + \hpsidag(2) \int d3 v(2,3) \hpsidag(3)\hpsi(3) \right]
\end{split}
\end{align}


		