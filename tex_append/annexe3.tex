\section{Computational parameters for Chapter 2} \label{app:comp_par_strain}
%
We applied different strains along the x-direction, the one parallel to the B-N bond, ranging from $-2.5\%$ to $+2.5\%$ of the equilibrium cell vector. Then we allowed the cell vectors and atomic positions to relax only in the other two  orthogonal directions, keeping the arbitrarily strained length fixed. The relaxation is done using Density Functional Theory and a damped molecular dynamics algorithm as implemented in the QuantumEspresso code,\cite{giannozzi2009quantum} with norm-conserving pseudo-potentials in the Local Density Approximation (LDA), a kinetic energy cutoff of 120 Ry and an equivalent Monkhorst-Pack grid of $18 \times 18 \times 6$ $k$-points. The forces acting on the cell and atoms were converged to be lower than $10^{-6}$ a.u.

Once the strained orthorhombic cells were relaxed, we have constructed strained pseudo-hexagonal cells in order to proceed with the electronic and optical calculations.

On these pseudo-hexagonal 2-atom cells we performed phonon calculations using \acrshort{DFPT},\cite{giannozzi2009quantum} with $\qq$-points and $\kk$-sampling respectively of $6 \times 6 \times 2$ and $18 \times 18 \times 6$, in order to verify the stability of our structure and the effect of strain on phonon modes.

The quasi-particle band structure was obtained within the G$_0$W$_0$ approximation, using again a $18 \times 18 \times 6$ k-points sampling, with 210 bands plus a terminator\cite{bruneval2008accurate} for G and W in order to speed up convergence. We used a cutoff of 7 Ha for the dielectric constant that was calculated within the plasmon-pole approximation. Excitons and optical absorption were studied solving the Bethe-Salpeter equation\cite{strinati1988application} using 4 valence and 4 conduction bands, as implemented in the \yambo code,\cite{Sangalli_2019} using the same k-points grid as for the G$_0$W$_0$ calculations. 

Luminescence was calculated following the approach described in Ref. \cite{paleari2019exciton}. We searched for the minima of the indirect gap within the independent particle approximation and we used the corresponding $\qq$-vectors to construct a supercell that map these points at $\Gamma$. We displaced atoms along all possible phonon modes having a periodicity commensurate with the different $\qq$-points of the indirect gap minima, and calculated the derivatives of the excitonic optical matrix elements.\footnote{In the luminescence calculations we used a smaller k-point sampling, $12 \times 12 \times 4$ and a scissor operator to speed up calculations in the supercells, similar to Refs.~\cite{cannuccia2019theory,paleari2019exciton}. These parameters are sufficient to describe the lowest exciton that is responsible for the luminescence.}  With these ingredients plus the phonon frequencies we reconstructed the spectra using the van Roosbroeck-Shockley (RS) relation. \cite{paleari2019exciton}