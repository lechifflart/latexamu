\chapter{Ab initio exciton-phonon coupling} \label{chap:mBN}
\textit{This Chapter is partly based on our publication Ref. \cite{lechifflart2023first}. Some of the text and figures contained in this Chapter are adapted from this reference.}


\chaptertoc{}
%


%
\section{Introduction}
In recent years, single or few-layer materials have attracted a great deal of attention due to their peculiar properties, often different from their bulk counterparts. For example, MoS$_2$ undergoes an indirect-to-direct band gap transition when reducing its thickness to the monolayer limit.\cite{splendiani2010emerging,Mak_2010}
This transition was discovered thanks to the increase in the luminescence signal, since it is well-known that indirect materials tend to be poor light emitters due to higher-order processes mediating the electron-hole recombination.
A similar band gap transition was also predicted for \acrshort{hBN}.\cite{paleari2018excitons}

For many years it was not possible to measure the luminescence signal of a single hBN layer,\cite{schue2016dimensionality} and this was attributed either to the increase of the exciton-exciton annihilation rate in low-dimensional structures\cite{yuan2015exciton,plaud2019exciton} or to other quenching mechanisms.
However, recent experiments reported a luminescence signal from direct excitons in single-layer hexagonal Boron Nitride (mBN) epitaxially grown on Graphite, showing the existence of a fine two-peak structure.\cite{elias2019direct,wang2022scalable} These experiments were later repeated using exfoliated hBN on a silicon oxide substrate,\cite{rousseau2021monolayer} where only one dominant peak was found. Very recently, a group achieved the technical prowess of measuring the cathodoluminescence signal of a monolayer of hBN grown on a Graphite substrate.\cite{shima2023cathodoluminescence} Their results have a very low signal-to-noise ratio, but it seems that only one peak appears.
The various mBN spectra that appeared in the literature present notable differences which were attributed first to coupling with phonon modes and later to the presence of bubbles in the m-hBN structure.
In addition, the first luminescence measurements of another polytype of BN with an AB stacking, the so-called \acrfull{bBN} was reported recently.\cite{rousseau2022bernal, rousseau2022phonon}
These measurements seemed to show the coexistence of emission peaks from both direct and indirect excitons in the same spectrum.
From a theoretical point of view, mBN has been always considered a direct band gap materials in models\cite{galvani2016excitons}, while the nature of its gap in \emph{ab initio} approaches depends on the approximation used in the calculations.\cite{prete2020giant,mengle2019impact} Regarding bulk hBN, models and \emph{ab initio} calculations agree on its nature as an indirect gap insulator.\cite{sponza2018direct}
For the intermediate situation, for few-layers hBN, the magnitude and nature of the quasiparticle band gap depends both on the number of layers and on the stacking order.\cite{sponza2018direct,mengle2019impact,latil2022electronic}

In light of these results, we decided to investigate luminescence in mBN using a novel approach that includes the coupling between excitons and phonons within an \emph{ab initio} framework and allows for an accurate treatment of both direct and phonon-assisted peaks in the spectra.
The motivation of this study is threefold.
First, mBN could present both direct and indirect peaks in its luminescence spectra, which is an ideal test for our theory, while its well-known bulk counterpart provides an excellent benchmark.
Second, the presence of new and partially unclear experiments on \acrshort{mBN} makes the application of this new methodology interesting and timely.
Third, a detailed study of the relation between the lattice structure and phonon and exciton dispersions could pave to way to an experimental tuning of the intensity of various features in the luminescence spectra.  

At variance with older theoretical works on exciton-phonon coupling, where the values of the coupling matrix elements were taken as parameters, recent formulations focused on accurate \emph{ab initio} numerical simulations, either tackling the exciton-phonon problem by means of finite-difference displacements in supercells as was done in the last chapter, or, more recently, by combining \acrshort{DFPT} with \acrshort{BSE} simulations, in order to avoid the need of large supercells.\cite{chen2020exciton}
In the first two sections of this chapter we put forward a formal derivation within \acrshort{MBPT} which captures phonon mediated photoluminescence in a steady-state approximation, and combine it with \acrshort{DFPT} to perform accurate \emph{ab initio} numerical simulations in the primitive unit cell.
The great advantage of this formulation is the possibility of integrating over exciton momenta in the full Brillouin Zone, thus calculating the renormalization of the direct peak induced by the indirect transitions. This is essential when studying an emission spectrum that may have competing direct and indirect peaks, such as the case investigated here. We test this method on the well-documented bulk \acrshort{hBN} and then apply it to \acrshort{mBN}, which constitute the main result of this Chapter. Finally, we present preliminary results on the \acrshort{bBN}.

%
\section{Theory of the ab initio exciton-phonon coupling} \label{sec:excph_ai}
In this section we present an \textit{ab initio} approach to obtain the exciton-phonon coupling matrix elements, that goes beyond the finite-difference approach presented in Sec. \ref{sec:excph_fdd}. The great advantage of this formulation is the possibility to integrate over exciton momenta in the full \acrlong{BZ} and to obtain the coupling between all excitons and all phonon modes.\\

For this work we based our numerical implementation on the formula derived by our collaborator Fulvio Paleari in his PhD thesis, \cite{paleari2019first} itself stemming from the theoretical work of Pierluigi Cudazzo published in Ref. \cite{cudazzo2020first}. In this approach, the electron-phonon interaction is included in the \acrshort{BSE} kernel via a phonon propagator. It induces a dynamic perturbation of the static electron-hole interaction contained in the kernel. It gives rise to a dynamical \acrlong{BSE}. Then, after a few approximations, the problem can be formally inverted and mapped onto an exciton-phonon Hamiltonian, which gives the exciton-phonon matrix elements. I refer the interested reader to these two references \cite{paleari2019first,cudazzo2020first}, where the rigorous derivation can be found.\\

In this thesis, I will present another way to derive the same exciton-phonon matrix elements, adapted from Ref. \cite{chen2020exciton}. It uses first-order perturbation theory for the excitonic Hamiltonian, the perturbation being a displacement of atoms along phonon modes. It introduces an additional interaction term due to the electron-phonon coupling, from which the exciton-phonon matrix elements can be identified. \\
Note that another, more general approach exists in litterature. It consists in treating the electron-electron, the electron-phonon interactions and the external field on the same footing,\cite{paleari2022exciton} which lifts some of the approximation we use, but it does not introduce relevant changes for the systems investigated here.\\

We consider a system with displacements from equilibrium positions $\boldsymbol{u}_{Ls}$ ($L$ labels the unit cell and $s$ the atom). We start from the \acrshort{DFT} level and take the Taylor expansion of the Kohn-Sham potential, labelled as $v_{\text{eff}}$ in Eq. \eqref{eq:KS_potential}, which we call $V^{KS}$ here. The expansion around the equilibrium positions reads :
\begin{equation}
    V^{KS}(\left\{ \uu_{Ls} \right\}) = V_0^{KS} + \sum_{Ls\alpha} \frac{\partial V^{KS}}{\partial \uu_{Ls\alpha}} \uu_{Ls\alpha} + \mathcal{O}(\left\{ \uu_{Ls} \right\}^2)
\end{equation}
The electronic wave functions and eigenvalues of the perturbed system depend on the atomic displacements $\left\{ \uu_{Ls} \right\}$. To obtain their change in the perturbed system, we apply first-order perturbation theory by keeping terms linear in $\left\{ \uu_{Ls} \right\}$. To first order, the correction to the eigenvalues vanishes while the correction to the Kohn-Sham wave functions $\psi_i$ (solutions of Eq. \eqref{eq:KS_eqs}) can be written as : 
\begin{equation}
    \delta\ket{\psi_i} = \sum_{j\neq i} \frac{\bra{\psi_j} \Delta V \ket{\psi_i}}{\epsilon_i - \epsilon_j} \ket{\psi_j}, \qquad \text{with} \ \ \Delta V = \sum_{Ls\alpha}\frac{\partial V^{KS}}{\partial \uu_{Ls}}\cdot \uu_{Ls}
\end{equation}
In the following, we use the tilde to label quantities of the perturbed system and write the perturbed wave function as :
\begin{equation}
    \ket{\tilde{\psi}_i} = \ket{\psi_i} + \delta\ket{\psi_i} = \ket{\psi_i} + \sum_{j \neq i} \Delta_{ij} \ket{\psi_j} \label{eq:perturb_wf}
\end{equation}
with
\begin{equation}
    \Delta_{ij} \equiv \frac{\bra{\psi_j} \Delta V \ket{\psi_i}}{\epsilon_i - \epsilon_j} \label{eq:Delta_var}
\end{equation}

We set ourselves in the Tamm-Dancoff approximation and we use the resonant Hamiltonian from Eq. \eqref{eq:H_BSE_res} as the  Hamiltonian of the unperturbed system $H \equiv H^{2p}(\{\boldsymbol{u}_{Ls} \}=0)$. For the perturbed system, we have $\Tilde{H} \equiv H^{2p}(\{\boldsymbol{u}_{Ls} \})$. 
The perturbed Hamiltonian matrix element is :
\begin{equation}
    \Tilde{H}_{\Tilde{v}\Tilde{c},\Tilde{v}'\Tilde{c}'}  = \bra{\Tilde{v}\Tilde{c}} \tilde{H} \ket{\Tilde{v}'\Tilde{c}'}  = ( \Tilde{\epsilon}_c - \Tilde{\epsilon}_v ) \delta_{\Tilde{v}\Tilde{v}'}\delta_{\Tilde{c}\Tilde{c}'} + \tilde{\Xi}_{\tv\tc}^{\tv'\tc'} \label{eq:perturbed_BSE}
\end{equation}
where $v,c$ refer to valence and conduction bands, respectively.
The perturbed Bethe-Salpeter kernel is defined just as in Eq. \eqref{eq:BSE_kernel_vc}, except that it is evaluated with the screened interaction of the perturbed system $\tilde{W}$, and its matrix elements are expressed in the perturbed basis.

Solving the \acrshort{BSE} in Eq. \eqref{eq:BSE_secular} gives the exciton wave functions that we will name $\ket{\lambda}$ and energies $E_{\lambda}$ :
\begin{align}
    \sum_{v',c'} H_{vc,v'c'} A_{\lambda}^{v'c'} &= E_{\lambda} A_{\lambda}^{vc} \\
    \ket{\lambda} = \sum_{vc}& A^{vc}_{\lambda}\ket{vc}
\end{align}
To derive the exciton-phonon interaction, we project the perturbed BSE Hamiltonian onto the unperturbed basis set and keep only the terms to first-order in the phonon perturbation. 
By such a process, the terms that will arise and be different from the unperturbed BSE Hamiltonian will define the exciton-phonon interaction.
One can show that to first order, the perturbed and unperturbed electronic energies coincide, so we will use $\Tilde{\epsilon}_i = \epsilon_i$. 
The perturbed Hamiltonian in the unperturbed basis is :
\begin{align*}
    \Tilde{H}_{\lambda\lambda'} = \bra{\lambda'} \Tilde{H} \ket{\lambda} &= \sum_{\Tilde{v}\Tilde{c},\Tilde{v}'\Tilde{c}'}  \bra{\lambda'}\ket{\Tilde{v}\Tilde{c}}\bra{\Tilde{v}\Tilde{c}} \Tilde{H} \ket{\Tilde{v}'\Tilde{c}'}\bra{\Tilde{v}'\Tilde{c}'}\ket{\lambda} \\
    &= \sum_{vc,v'c'}\sum_{\Tilde{v}\Tilde{c},\Tilde{v}'\Tilde{c}'} \bra{\lambda'}\ket{vc}\bra{vc}\ket{\Tilde{v}\Tilde{c}}\bra{\Tilde{v}\Tilde{c}} \Tilde{H} \ket{\Tilde{v}'\Tilde{c}'}\bra{\Tilde{v}'\Tilde{c}'}\ket{v'c'}\bra{v'c'}\ket{\lambda}
\end{align*}
where we used the completeness relations of both basis sets, $\sum_{vc}\ket{vc}\bra{vc} = \mathds{1}$ and $\sum_{\Tilde{v},\Tilde{c}} \ket{\Tilde{v}\Tilde{c}}\bra{\Tilde{v}\Tilde{c}} = \mathds{1}$. By definition of the BSE wave functions $\bra{v'c'}\ket{\lambda} = A_{\lambda}^{v'c'}$, then we can write the above equation as :
\begin{equation}
    \Tilde{H}_{\lambda\lambda'} = \bra{\lambda'} \Tilde{H} \ket{\lambda} = \sum_{vc,v'c'} A_{\lambda'}^{vc*} A_{\lambda}^{v'c'} \times \left[ \sum_{\Tilde{v}\Tilde{c},\Tilde{v}'\Tilde{c}'} \bra{vc}\ket{\Tilde{v}\Tilde{c}}  \bra{\Tilde{v}\Tilde{c}} \Tilde{H} \ket{\Tilde{v}'\Tilde{c}'} \bra{\Tilde{v}'\Tilde{c}'}\ket{v'c'} \right]
    \label{eq:exc_ham}
\end{equation}
The term inside the square brackets can be separated in two :
\begin{align*}
    &\sum_{\Tilde{v}\Tilde{c},\Tilde{v}'\Tilde{c}'} \bra{vc}\ket{\Tilde{v}\Tilde{c}}  \bra{\Tilde{v}\Tilde{c}} \Tilde{H} \ket{\Tilde{v}'\Tilde{c}'} \bra{\Tilde{v}'\Tilde{c}'}\ket{v'c'} =  \sum_{\Tilde{v}\Tilde{c},\Tilde{v}'\Tilde{c}'} \bra{vc}\ket{\Tilde{v}\Tilde{c}}  \left[ (\Tilde{\epsilon}_{\Tilde{c}} - \Tilde{\epsilon}_{\Tilde{v}}) \delta_{\Tilde{v}\Tilde{v}'}\delta_{\Tilde{c}\Tilde{c}'} + \tilde{\Xi}_{\tv\tc}^{\tv'\tc'} \right] \bra{\Tilde{v}'\Tilde{c}'}\ket{v'c'} \\
    &= \sum_{\Tilde{v}\Tilde{c}} \bra{vc}\ket{\Tilde{v}\Tilde{c}} (\epsilon_{\Tilde{c}} - \epsilon_{\Tilde{v}}) \bra{\Tilde{v}\Tilde{c}}\ket{v'c'} + \sum_{\Tilde{v}\Tilde{c},\Tilde{v}'\Tilde{c}'} \bra{vc}\ket{\Tilde{v}\Tilde{c}} \tilde{\Xi}_{\tv\tc}^{\tv'\tc'} \bra{\Tilde{v}'\Tilde{c}'}\ket{v'c'}
\end{align*}
We make the choice to approximate the perturbed kernel with the unperturbed one, $\tilde{\Xi}_{\tv\tc}^{\tv'\tc'} \approx \bra{\Tilde{v}\Tilde{c}} \Xi \ket{\Tilde{v}'\Tilde{c}'}$, that is to say the effect of atomic displacements on the screened interaction can be neglected and $W \approx \Tilde{W}$. This is the same approximation we took in Chapter 2 when we evaluated the response function by finite difference derivative. With this approximation we have :
\begin{equation}
    \sum_{\Tilde{v}\Tilde{c},\Tilde{v}'\Tilde{c}'} \bra{vc}\ket{\Tilde{v}\Tilde{c}} \tilde{\Xi}_{\tv\tc}^{\tv'\tc'} \bra{\Tilde{v}'\Tilde{c}'}\ket{v'c'} 
    \approx  \sum_{\Tilde{v}\Tilde{c},\Tilde{v}'\Tilde{c}'} \bra{vc}\ket{\Tilde{v}\Tilde{c}} 
    \bra{\Tilde{v}\Tilde{c}} \Xi \ket{\Tilde{v}'\Tilde{c}'}\bra{\Tilde{v}'\Tilde{c}'}\ket{v'c'} = \bra{vc} \Xi \ket{v'c'} = \Xi_{vc}^{v'c'}
\end{equation}
and thus the term in square brackets in Eq. \eqref{eq:exc_ham} becomes 
\begin{equation}
    \sum_{\Tilde{v}\Tilde{c},\Tilde{v}'\Tilde{c}'} \bra{vc}\ket{\Tilde{v}\Tilde{c}}  \bra{\Tilde{v}\Tilde{c}} \Tilde{H} \ket{\Tilde{v}'\Tilde{c}'} \bra{\Tilde{v}'\Tilde{c}'}\ket{v'c'} = \sum_{\Tilde{v}\Tilde{c}} \bra{vc}\ket{\Tilde{v}\Tilde{c}} (\epsilon_{\Tilde{v}} - \epsilon_{\Tilde{c}}) \bra{\Tilde{v}\Tilde{c}}\ket{v'c'} +  \Xi_{vc}^{v'c'} 
\end{equation}
Next, we use Eq. \eqref{eq:perturb_wf} to expand $\sum_{\Tilde{v}\tilde{c}} \bra{vc}\ket{\Tilde{v}\Tilde{c}} (\epsilon_{\Tilde{c}} - \epsilon_{\Tilde{v}})\bra{\Tilde{v}\Tilde{c}}\ket{v'c'}$ to order $\mathcal{O}(\Delta)$. We work within the Tamm-Dancoff approximation and keep only the resonant part of the BSE Hamiltonian; as a consequence, only valence-valence and conduction-conduction phonon-mediated scattering are allowed, that is to say $\Delta_{vc} = \Delta_{cv} = 0$ where the operator $\Delta$ was defined in Eq. \eqref{eq:Delta_var}. Using Eq. \eqref{eq:perturb_wf} we get :
\begin{align}
\begin{split}
    \bra{vc}\ket{\tv\tc} &= \bra{v}\ket{\tv}\bra{c}\ket{\tc} = \left(\delta_{v\tv} + \sum_{v''\neq\tv}\Delta_{\tv v''}\delta_{vv''}\right) \left(\delta_{c\tc} + \sum_{c''\neq \tc}\Delta_{\tc c''}\delta_{cc''}\right) \\
    &= \left( \delta_{v\tv}\delta_{c\tc} + \delta_{v\tv} \sum_{c''\neq\tc} \Delta_{\tc c''}\delta_{cc''} + \delta_{c\tc}\sum_{v''\neq \tv} \Delta_{\tv v''} \delta_{vv''} \right) + \mathcal{O}(\Delta^2)
\end{split}
\end{align}
and similarly
\begin{equation}
    \bra{\tv\tc}\ket{v'c'} = \bra{v'}\ket{\tv}^*\bra{c'}\ket{\tc}^* = \left( \delta_{v'\tv}\delta_{c'\tc} + \delta_{v'\tv} \sum_{c''\neq\tc} \Delta^*_{\tc c''}\delta_{c'c''} + \delta_{c'\tc}\sum_{v''\neq \tv} \Delta^*_{\tv v''} \delta_{v'v''}  \right)  + \mathcal{O}(\Delta^2)
\end{equation}
With these expressions, there are five first-order terms in $\sum_{\Tilde{v}\tilde{c}} \bra{vc}\ket{\Tilde{v}\Tilde{c}} (\epsilon_{\Tilde{c}} - \epsilon_{\Tilde{v}})\bra{\Tilde{v}\Tilde{c}}\ket{v'c'}$ that we can simplify using the Kronecker delta :
\begin{align}
    \sum_{\tv\tc} \bra{vc}\ket{\tv\tc} (\epsilon_{\Tilde{c}} - \epsilon_{\Tilde{v}})\bra{\tv\tc}\ket{v'c'} \nonumber \\
    %
        \approx (\epsilon_c - \epsilon_v)\delta_{vv'}\delta_{cc'} &+ \delta_{cc'}\sum_{\tv}(\epsilon_c - \epsilon_{\tv}) \sum_{v''\neq \tv} (\Delta^*_{\tv v''}\delta_{vv''}\delta_{v'\tv} + \Delta_{\tv v''}\delta_{v'v''}\delta_{v\tv}) \nonumber \\
        &+  \delta_{vv'} \sum_{\tc} (\epsilon_{\tc} - \epsilon_v) \sum_{c'' \neq\tc} (\Delta_{\tc c''}\delta_{cc''}\delta_{c'\tc} + \Delta^*_{\tc c''}\delta_{c'c''}\delta_{c\tc}) \nonumber \\
        %
        = (\epsilon_c - \epsilon_v)\delta_{vv'}\delta_{cc'} &+ \delta_{cc'} \left[  \sum_{v''\neq v'} (\epsilon_c - \epsilon_{v'}) \Delta^*_{v'v''}\delta_{v v''} + \sum_{v'' \neq v} (\epsilon_c - \epsilon_v) \Delta_{vv''} \delta_{v'v''} \right] \nonumber \\
        &+  \delta_{vv'} \left[ \sum_{c''\neq c'} (\epsilon_{c'} - \epsilon_v)\Delta_{c' c''}\delta_{cc''} + \sum_{c''\neq c} (\epsilon_c - \epsilon_v) \Delta^*_{cc''}\delta_{c'c''} \right] \nonumber \\
    %
    = (\epsilon_c - \epsilon_v) \delta_{vv'}\delta_{cc'} &+ \delta_{cc'}(\epsilon_{v'} - \epsilon_v) \Delta_{vv'} + \delta_{vv'}(\epsilon_c - \epsilon_{c'}) \Delta^*_{cc'}
\end{align}
where we used $\Delta_{ij} = -\Delta_{ji}^*$ to obtain the last line. Finally, the perturbed Hamiltonian in the excitonic basis in Eq. \eqref{eq:exc_ham} becomes :
\begin{align}
    \tilde{H}_{\lambda\lambda'} &= \sum_{vc,v'c'} A_{\lambda'}^{vc*} A_{\lambda}^{v'c'} \times \left\{ \left[ (\epsilon_c - \epsilon_v) \delta_{vv'}\delta_{cc'} + \Xi_{vc}^{v'c'} \right] + \delta_{cc'} (\epsilon_{v'} - \epsilon_v)\Delta_{vv'} + \delta_{vv'}(\epsilon_c - \epsilon_{c'}) \Delta_{cc'}^*  \right\} \nonumber \\
    &= E_{\lambda'}\delta_{\lambda\lambda'} + \sum_{vc,v'c'} A_{\lambda'}^{vc*} A_{\lambda}^{v'c'} \cdot \left( \delta_{cc'}(\epsilon_{v'} - \epsilon_v) \Delta_{vv'}  + \delta_{vv'} (\epsilon_c - \epsilon_{c'}) \Delta^*_{cc'} \right) \label{eq:perturb_H_exc}
\end{align}
where we use the fact that the unperturbed Hamiltonian is diagonalized by the Tamm-Dancoff exciton eigenvectors :
\begin{equation}
    E_{\lambda'}\delta_{\lambda\lambda'} = \sum_{vc,v'c'} A_{\lambda'}^{vc*} A_{\lambda}^{v'c'} \times \left( (\epsilon_c - \epsilon_v)\delta_{vv'}\delta_{cc'} + \Xi_{vc}^{v'c'} \right).
\end{equation}
Therefore, the first term in the second line of Eq. \eqref{eq:perturb_H_exc} is the unperturbed Hamiltonian, while the second term is the exciton-phonon interaction,
\begin{equation}
    \tilde{H}_{\text{exc-ph}} = \sum_{vc,v'c'} A_{\lambda'}^{vc*} A_{\lambda}^{v'c'} \cdot \left( \delta_{cc'}(\epsilon_{v'} - \epsilon_v) \Delta_{vv'}  + \delta_{vv'} (\epsilon_c - \epsilon_{c'}) \Delta^*_{cc'} \right). \label{eq:H_exc-ph}
\end{equation}
To obtain the final result, we reintroduce the periodicity of the Kohn-Sham states stemming from Bloch theorem :
\begin{equation}
    \ket{\phi_i} \to \ket{\phi_{n\kk}}
\end{equation}
and the transition basis set for an exciton with center of mass momentum $\QQ$ is $\ket{vc} = \ket{v\kk_v,c\kk_c} = \ket{v\kk_v,c\kk_v + \QQ}$. We write the change in potential due to atomic displacements in second quantization using the phonon normal coordinates :
\begin{equation}
    \Delta V = \sum_{\mu \qq} \sqrt{\frac{1}{2\Omega_{\mu\qq}}} \partial_{\mu\qq} V^{KS}(\hat{b}_{\mu\qq} + \hat{b}^\dagger_{\mu-\qq})  
\end{equation}
where the operator $\partial_{\mu\qq}$ should be understood as the derivative with respect to a displacement $R_{\mu\qq}$ along a phonon mode $\mu$ at momentum $\qq$. Then the $\Delta_{ij}$ describing the transition from $i$-th to $j$-th state becomes :
\begin{equation}
    \Delta_{n\kk n'\kk'} = \frac{\bra{n'\kk'} \Delta V \ket{n\kk}}{\epsilon_{n\kk} - \epsilon_{n'\kk'}} = \sum_{\mu\qq} \frac{g_{nn'\mu}(\kk,\qq) \delta(\kk'-\kk-\qq)}{\epsilon_{n\kk} - \epsilon_{n'\kk'}} (\hat{b}_{\mu\qq} + \hat{b}^\dagger_{\mu-\qq})
\end{equation}
where $g_{nn'\mu}(\kk,\qq)$ are the electron-phonon matrix elements defined in Eq. \eqref{eq:gkkp}, namely the probability amplitude for an electron in band $n$ with crystal momentum $\kk$ to transition to a final state in band $n'$ and momentum $\kk' = \kk+\qq$ by absorbing or emitting a phonon with mode index $\mu$ and wave vector $\qq$. The slight difference with Eq. \eqref{eq:gkkp} is that we change the arguments to have a more compact form : the first argument is the momentum of the initial state and the second argument is the phonon momentum.

To proceed, we have to make the approximation that the excitons are boson-like particles bound by the Coulomb interaction. We can then define a bosonic Hamiltonian for excitons :
\begin{equation}
    H_{\text{exc}} = \sum_\lambda E_\lambda (\QQ) \hat{a}^\dagger_{\lambda\QQ} \hat{a}_{\lambda\QQ}
\end{equation}
where $\hat{a}^\dagger_{\lambda\QQ}, \hat{a}_{\lambda\QQ}$ are the creation/annihilation operators for an exciton $\lambda$ with center-of-mass momentum $\QQ$ and energy $E_\lambda(\QQ)$. This is an approximation that ignores the fact that excitons are a pair of two bound fermions, this is why it works best at low exciton density so that the exciton are weakly interacting. It has been shown to correctly reproduce several experimental results.\cite{paleari2019exciton,perebeinos2005effect} However, there are theoretical evidence that the fermionic character of excitons cannot always be neglected.\cite{katzer2023excitonphononscattering}\\
Using this approximation we rewrite the exciton-phonon interaction from Eq. \eqref{eq:H_exc-ph} in second quantization :
\begin{equation}
    \tilde{H}_{\text{exc-ph}} = \sum_{\lambda\lambda'\mu, \QQ\qq} \mathcal{G}_{\lambda\lambda'}^{\mu}(\QQ,\qq) \hat{a}^\dagger_{\lambda\QQ+\qq} \hat{a}_{\lambda'\QQ} (\hat{b}_{\mu\qq} + \hat{b}^\dagger_{\mu-\qq}). \label{eq:H_excph}
\end{equation}
where we defined the exciton-phonon matrix elements as :
\begin{multline}
    \mathcal{G}_{\lambda\lambda'}^{\mu}(\QQ,\qq) = \sum_{\substack{vcv'c'\\ \kk_v \kk_c \kk'_{v'} \kk'_{c'}}} A_{\lambda\QQ+\qq}(v\kk_v,c\kk_c) A_{\lambda'\QQ}^*(v'\kk'_{v'},c'\kk'_{c'}) \\ 
    \times \left[ \delta_{vv'} g_{c'c\mu}(\kk'_{c'},\qq) \delta(\kk_c - \kk'_{c'} - \qq) \right. \left. - \delta_{cc'}g_{vv'\mu}(\kk_v,\qq) \delta(\kk'_{v'} - \kk_v -\qq) \right]. \label{eq:Gkkp_full}
\end{multline}
Let us make momentum conservation explicit to obtain the final expression. The exciton-phonon coupling constant $\mathcal{G}_{\lambda\lambda'\mu}(\QQ,\qq)$ is the probability amplitude for scattering from an exciton with band index $\lambda$ with center-of-mass momentum $\QQ + \qq$ to an exciton with band index $\lambda'$ and center-of-mass momentum $\QQ$. This convention will be clarified later. Since $A_{\lambda\QQ}(v\kk_{v},c\kk_{c}) \neq 0$ only for $\kk_c - \kk_v = \QQ$, in Eq. \eqref{eq:Gkkp_full} we can impose three constraints : $\kk_c - \kk_v = \QQ$, $\kk'_c - \kk'_v =\QQ + \qq$ and $\kk'_c - \kk_c = \qq$ (or $\kk'_v - \kk_v = \qq$). As a consequence, we drop three $\kk$-point \acrshort{BZ} summations and the final result for the exciton-phonon matrix element for a given exciton momentum $\QQ$ and phonon momentum $\qq$ is :
\begin{multline}
    \mathcal{G}_{\lambda\lambda'}^{\mu}(\QQ,\qq) \\
    = \sum_{\kk} \left[ \sum_{vcc'} A_{\lambda\QQ+\qq}(v\kk,c\kk+\QQ+\qq) A_{\lambda'\QQ}^{*}(v\kk,c'\kk+\QQ)g_{c'c\mu}(\kk+\QQ +\qq,-\qq) \right.\\
     \left. - \sum_{cvv'} A_{\lambda\QQ+\qq}(v\kk-\qq,c\kk+\QQ) A_{\lambda'\QQ}^*(v'\kk,c\kk+\QQ) g_{vv'\mu}(\kk-\qq,\qq) \right]. \label{eq:Gkkp}
\end{multline}
This is the general expression which was implemented in the \yambo~code. This expression is made of two contributions relative to the coupling of phonons with either the electron in the conduction band or the hole in the valence band constituting the exciton. It corresponds to a rotation of the electron-phonon coupling in the exciton basis. This basis shifts the picture from electrons and holes scattering with phonons to transitions between excitonic states mediated by phonons.

This formula allows to compute various quantities depending on the exciton-phonon matrix elements, such as the exciton lifetimes or more interestingly for us, the response function that includes phonon-assisted transitions. This will allow us to compute luminescence spectra. 

%
\section{Phonon-assisted response function}
Now we can proceed to the solution of the exciton-phonon Hamiltonian. A direct diagonalization is out of reach because the transitions at different $\QQ$ are mixed by the electron-phonon scattering, therefore the dimension of the Hilbert space becomes too large. Hence we will make use of \acrshort{MBPT} to find an approximate solution to Eq. \eqref{eq:H_excph}. The dynamical perturbation induced by the electron-phonon coupling adds a term to the \acrshort{BSE} kernel and it yields a general \textit{dynamical} \acrlong{BSE} :
\begin{equation}
    \mathcal{L}(1234) = L(1234) + \int d5678 \ L(1625) \ \tilde{\Xi}^{D}(5867) \ \mathcal{L}(7483) \label{eq:dBSE}
\end{equation}
where $L$ is the two-particle propagator solution of the static \acrshort{BSE} in Eq. \eqref{eq:BSE} and the kernel $\tilde{\Xi}^{D}$ has an additional dynamical term induced by the electron-phonon interaction. The dynamical kernel does not allow a direct inversion of the dynamical \acrshort{BSE} since it depends self-consistently on $L$ and cannot be written in terms of two times. We chose an approach that consists in taking the electron-phonon interaction only up to first order. This way, we define $L^{(1)}$ as the solution of Eq. \eqref{eq:dBSE}, obtained by replacing $\mathcal{L}$ on the right hand side of Eq. \ref{eq:dBSE} by the static $L$ :
\begin{equation}
    L^{(1)}(1234) = L(1234) + \int d5678 \ L(1625) \ \tilde{\Xi}(5867) \ L(7483)
\end{equation}
where $\Tilde{\Xi}$ is the kernel perturbed by first-order electron-phonon interaction, the same introduced in the above section in Eq. \eqref{eq:perturbed_BSE}. With this equation can consider only interaction mediated with a single phonon. Extension to multiple-phonon scattering exists in literature,\cite{perebeinos2008phonon} but the present formulation is the first order of a cumulant expansion and its generalization to coupling with phonons at all orders is straightforward.\cite{cudazzo2020first}\\
Using the relation Eq. \eqref{eq:chi_iL}, we can obtain the response function in the excitonic basis in terms of one frequency (or two times), including the first-order correction due to exciton-phonon coupling :
\begin{equation}
    \chi^{(1)}_{\lambda\lambda'}(\omega) = \chi_{\lambda\lambda'}(\omega) + \chi_{\lambda}(\omega) \ \Pi^{\text{excp-ph}}_{\lambda\lambda'} (\omega) \ \chi_{\lambda'}(\omega) \label{eq:chi_1}
\end{equation}
where we used the short-hand notation $\chi_{\lambda}(\omega) = \chi_{\lambda\lambda'} (\omega) \delta_{\lambda\lambda'}$. On the right hand side, the quantity $\Pi^{\text{excp-ph}}_{\lambda\lambda'} (\omega)$ is the exciton self-energy describing dynamical effects induced by the electron-phonon interaction. We will refer to it as \textit{exciton-phonon self-energy}. It can be computed explicitly in \acrshort{MBPT}\cite{mahan2000many} from the perturbed kernel $\tilde{\Xi}$ in the Tamm-Dancoff approximation, and we do it in a similar way as in Ref. \cite{giustino2017review} for the electron-phonon case. Owing to the fact that we consider the first order only in exciton-phonon scattering, we will have a self-energy similar to the Fan-Migdal one for the electron-phonon problem : $\Pi^{\text{excp-ph}} = \mathcal{G}^2DL$ where the exciton propagator $L$ replaces the electron one, $D$ is the phonon propagator and $\mathcal{G}$ are the exciton-phonon matrix elements derived in the previous section given by Eq. \eqref{eq:Gkkp}. 

Keeping only the first order dynamical correction allows scattering with only one phonon, but the use of a self-energy allows to formally write two different ways of summing contributions to infinite orders, hence scattering with any number of phonons. The first one is to take a Dyson equation $\chi^{D} = \chi + \chi \Pi^{\text{excp-ph}} \chi$ which corresponds to a partial re-summation of the general Eq. \eqref{eq:dBSE}.\cite{marini2003dynamical} It will yield the correction to the exciton energies due to coupling with phonons but will fail to describe the phonon satellites in the optical spectra, just as the \acrshort{PES} satellites given by the $GW$ approximation are inaccurate. Another infinite summation is the cumulant expansion $\chi^C = \chi e^C$ with the cumulant coefficient of the form $C_\lambda(t) = \int dt \Pi^{\text{excp-ph}}_{\lambda\lambda}(t) e^{iE_\lambda t}$, determined by the diagonal components of the self-energy.\cite{cudazzo2020first} The cumulant ansatz is able to capture the physics giving rise to phonon satellites.\\
In fact, the first order of these two summations is identical and is given by Eq. \eqref{eq:chi_1}. It is the one we use for the rest of this thesis. We will focus on the description of the satellite structures in optical spectra and neglect the corrections to exciton energies. The exciton-phonon self-energy writes :
\begin{multline}
    \Pi^{\text{excp-ph}}_{\lambda\lambda'} (\QQ,\omega) = \frac{1}{N_q} \sum_{\mu\beta\qq} \mathcal{G}^\mu_{\beta\lambda}(\QQ,\qq) \mathcal{G}^{\mu*}_{\beta\lambda'}(\QQ,\qq) \\
    \times \left[ \frac{1-n_\beta(\QQ+\qq)+n_{\qq\mu}}{\omega-E_{\QQ+\qq,\beta} + \Omega_{\qq\mu} + i\eta} + \frac{n_\beta(\QQ+\qq)+n_{\qq\mu}}{\omega-E_{\QQ+\qq,\beta} - \Omega_{\qq\mu} + i\eta} \right] \label{eq:excph_SE}
\end{multline}
where $N_q$ is the number of $q$-points summed over in the Brillouin Zone, $\Omega_{\qq\mu}$ is the frequency of phonon mode $\mu$ at momentum $\qq$, $n_{\qq\mu}$ and $n_\beta(\QQ)$ are the temperature-dependent occupation factors for phonons and excitons, respectively. From here on, we label $\beta$ the finite-momentum, lowest lying excitons states that are populated. The internal sum over $\beta$ excitons includes every possible exciton level $E_{\QQ+\qq,\beta}$ that can be connected to the external exciton levels $E_{\lambda,\QQ}$ by emitting or absorbing one phonon with frequency $\Omega_{\qq\mu}$.

To simulate the process of luminescence, we assume that the sample is constantly pumped with a laser and that there is a quasi-equilibrium population of excited carriers. In the excitonic picture, it means that the minima of the exciton dispersion are populated. When these minima are at indirect momenta, the process of light emission will start from an exciton $\beta$ at finite momentum $\qq$ that will be scattered by a phonon with the same momentum $\qq$ and frequency $\Omega_{\qq}$ into a direct exciton $\lambda$ at $\QQ = 0$. This direct exciton is allowed to recombine radiatively but it is a virtual, intermediate state. The frequency of the emitted light will be $\hbar \omega_{PL} = \hbar E_{\qq\beta} \pm \hbar\Omega_{\qq\mu}$. This allows us to simplify the expression of the exciton-phonon matrix elements in Eq. \eqref{eq:Gkkp} as $\mathcal{G}_{\lambda\lambda'}^\mu(\QQ=0,\qq) = \mathcal{G}_{\lambda\lambda'}^\mu(\qq)$. Then, we make two approximations to compute the self-energy in Eq. \eqref{eq:excph_SE}. The first one is to neglect the excitonic occupations $n_\beta$ compared to the phonon ones $n_{\qq\mu}$. This is realistic for the quasi-equilibrium situation we are interested in (see Supplemental Materials of Ref. \cite{schue2019bright} for an estimation of the excitonic density in hBN). The second is to use only the diagonal components of the self-energy.\cite{toyozawa1964interband} \textcolor{red}{I didn't really understand what the off-diagonal terms do.} The simplified expression for the self-energy reads :
\begin{multline}
    \Pi^{\text{excp-ph}}_{\lambda}(\QQ=0,\omega) = \frac{1}{N_q} \sum_{\mu\beta\qq} |\mathcal{G}^\mu_{\beta\lambda}(\qq)|^2 \\
    \times \left[ \frac{1+n_{\qq\mu}}{\omega-E_{\qq\beta} - \Omega_{\qq\mu} + i\eta} + \frac{n_{\qq\mu}}{\omega-E_{\qq\beta} + \Omega_{\qq\mu} + i\eta} \right]. \label{eq:excph_SE_compact}
\end{multline}
Plugging Eq. \eqref{eq:excph_SE_compact} in Eq. \eqref{eq:chi_1}, we obtain for the diagonal case :
\begin{equation}
    \chi^{(1)}_\lambda(\omega) = \frac{|T^\lambda|^2 (1-R_\lambda)}{\omega - E_\lambda + i\eta} + \frac{|T^\lambda|^2}{N_q} \sum_{\mu\beta\qq} \frac{|\mathcal{G}^\mu_{\beta\lambda}(\qq)|^2}{(E_{\qq\beta} - E_\lambda \pm \Omega_{\qq\mu})^2} \frac{1/2 \pm 1/2 + n_{\qq\mu}}{\omega - E_{\qq\beta} \mp \Omega_{\qq\mu} + i\eta}
\end{equation}
where the upper (lower) sign refer to phonon emission (absorption). The above expression contains two terms : the first one describes the response coming from direct transitions. Its weight is reduced by the renormalization factor $R_\lambda$ compared to the static case. The second term gives the phonon satellites coming from transitions assisted by the absorption or emission of a single phonon. They appear at the energy of the finite-momentum excitons plus or minus on phonon frequency $E_{\qq\beta} \pm \Omega_{\qq\mu}$. The renormalization factor $R_\lambda$ is given by :
\begin{align}
    R_\lambda &= - \frac{\partial \Pi^{\text{excp-ph}}_{\lambda\lambda}(\omega)}{\partial \omega}\biggr|_{\omega = E_\lambda} \\
    &= \frac{1}{N_q} \sum_{\mu\beta\qq} |\mathcal{G}^\mu_{\beta\mu}(\qq)|^2 \left[ \frac{n_{\qq\mu} + 1}{(E_{\qq\beta} - E_\lambda + \Omega_{\qq\mu} )^2} + \frac{n_{\qq\mu}}{(E_{\qq\beta} - E_\lambda - \Omega_{\qq\mu})^2} \right] \label{eq:renorm_fact}
\end{align}
It is given by the derivative of the self-energy with respect to incoming light frequency. It is a measure of how much weight is transferred from the direct peak to the satellites structures in the optical spectra when including the exciton-phonon coupling. There is a divergence when the phonon frequency is resonant with the exciton energy difference in the denominators. This unphysical behavior is an artifact of the finite-order perturbation theory.\cite{toyozawa2003optical} If higher order terms are included, as in the cumulant expansion for instance, then a correction to the exciton energies and a broadening enter then denominators, which removes the divergence. In our case, one of the experimental hypothesis is that a phonon satellite caused by a transition between two zero-momentum excitons is visible in the photoluminescence spectrum of \acrshort{mBN}. Hence we have to take special care when performing the $q$ integration around $\Gamma$. To avoid divergences, we excluded the three acoustic modes with zero frequency at $\Gamma$ in the sum over phonon modes. Moreover to have a precise description of the phonon and exciton dispersions at small momentum, we used a double-grid scheme as explained in Appendix \ref{app:comp_details_Chapt3}.

Now we use Eq. \eqref{eq:eps_macro} to get the imaginary part of the macroscopic dielectric function, which in turn enters the van Roosbroeck--Shockley relations from Eqs. \eqref{eq:vRS_PL} and \eqref{eq:vRS_PL_ind}.
To include both the direct and the phonon-assisted emission, we use the sum 
\begin{equation}
	R^{sp}(\omega) = R^{sp}_0(\omega) + \frac{1}{N_q} \sum_{\qq\mu} R^{sp}_{\qq\mu}(\omega)
\end{equation} 
where $R^{sp}_0(\omega)$ is the spontaneous emission rate for direct transitions only given by Eq. \eqref{eq:vRS_PL} and the second term includes the phonon-assisted transition is given by Eq. \eqref{eq:vRS_PL_ind}. Note that here the refractive index entering the two term $n_1(\omega)$ can be excellently approximated by Eq. \eqref{eq:refrac_index} with just the static \acrshort{BSE} result for $\varepsilon^{exc}$. Our final luminescence intensity formula writes, up to a multiplicative constant :
\begin{multline}
    I^{\text{PL}}(\omega) = \mathcal{D} R^{sp}(\omega) = \mathcal{D} \Im \sum_\lambda |T^\lambda|^2 \left\{  \omega^3 n_1(\omega) \frac{1-R_\lambda}{\omega - E_\lambda + i\eta} \ e^{-\tfrac{E_\lambda-E_{min}}{k_B T_{exc}}} \right.\\
    +\left.  \frac{1}{N_q}\sum_{\mu\beta\qq} \omega(\omega \mp \Omega_{\qq\mu})^2 n_1(\omega) \frac{|\mathcal{G}^\mu_{\beta\mu}(\qq)|^2}{(E_{\qq\beta} - E_\lambda \pm \Omega_{\qq\mu} )^2} \frac{1/2 \pm 1/2 + n_{\qq\mu}}{\omega - E_{\qq\beta} \pm \Omega_{\qq\mu} + i\eta} \ e^{\tfrac{E_{\qq\beta}-E_{min}}{k_B T_{exc}}} \right\} \label{eq:I_PL}
\end{multline}
where $n_1(\omega)$ is the refractive index, $E_{min}$ is the minimum of the exciton dispersion and $T_{exc}$ is the effective excitonic temperature used to evaluate the Boltzmann occupation of the excitonic states. It is the only parameter in the whole process that needs to be fitted and we estimated it from the experimental measurements of Ref. \cite{cassabois2016hexagonal}. For a lattice temperature of 10 K, the fit gives an excitonic temperature of $T_{exc}=24$ K. The Boltzmann function for excitons is an approximation of the Bose-Einstein assumption that we made earlier, but it is valid for low excitonic density and reproduces correctly the experimental exponential decay of the phonon satellite peaks. We tried to use an occupation factor made out of the single-particle fermionic occupations rotated in the exciton basis, just as it is done in Refs. \cite{cannuccia2019theory,de2016unified,libbi2022phonon}, with an occupation function given by $n_{\lambda} = \sum_{cv\kk} \bra{\lambda}\ket{cv\kk} f_{c\kk} (1-f_{v\kk}) \bra{cv\kk}\ket{\lambda}$, where $f$ is the Fermi-Dirac function. However it gives rise to unphysical occupations as can be seen in Fig. \ref{fig:all_occup} for three different hBN materials. Indeed, the lowest excitonic state is populated, but so are the higher states coming from the same transitions, that are the analogous of the Wannier exciton excited states. 
\begin{figure}[h!b]
	\vspace{0.2cm}
	\setcapindent{2em}
	\centering
	\includegraphics[width=0.9\textwidth]{all_occupations.pdf}
	\caption{Comparison of Boltzmann (blue areas) and quasi-Fermi (red areas) excitonic occupations for \acrshort{mBN} (first column), \acrshort{hBN} (second column) and bBN (third column). See main text for the definition of the occupation functions and section \ref{sec:bulk_hBN} for the definition of the R$_1$ and R$_2$ points. The black lines are the Fourier interpolation of exciton dispersions, calculated at the G$_0$W$_0$+BSE level. \textcolor{blue}{Add a BZ w/ R1 \& R2, labels to distinguish the 3 materials.}}
	%MODIF Add a BZ w/ R1 \& R2, labels to distinguish the 3 materials.
	\label{fig:all_occup}
\end{figure}

The final expression for luminescence intensity Eq. \eqref{eq:I_PL} is to be compared with the one obtained with the finite difference method in Chapter 2, Eq. \eqref{eq:strain_vRS_PL}. Unlike the previous method where only the indirect transitions were included, in the present formula we have both the term coming from direct transitions and the term related to phonon satellites. The major theoretical advance here is that the renormalization factor from Eq. \eqref{eq:renorm_fact} allows to compare the relative intensities of the direct and the satellite peaks. Besides, the satellite energies include the addition or removal of the phonon frequency that arises from the dynamical correction in sesec:-phonon matrix elements. Thanks to this we can perform the whole workflow of \textit{ab initio} calculations in the unit cell, which is also an improvement from the previous method. 


%
\section{Excitonic properties of a monolayer of hexagonal Boron Nitride}
The electronic and optical properties of \acrlong{mBN} have been the subject of numerous studies using both \emph{ab initio} and semi-empirical methods.\cite{galvani2016excitons}
Within DFT, with the LDA exchange-correlation functionals, \acrshort{mBN} is a direct band gap material at high-symmetry point $K$, but the G$_0$W$_0$ corrections change its gap from direct to indirect, going from $K$ to $\Gamma$.\cite{prete2020giant} 
We verified that the system remains indirect even at the semi-self-consistent ``eigenvalue-GW'' level. We plot the electronic band structure in Fig. \ref{fig:mBN_bands_LDA_G0W0_G4W4} at different levels of theory : \acrshort{DFT} with the \acrshort{LDA}, G$_0$W$_0$ and evG$_4$W$_4$. The latter means that we iterated the Hedin's equation in the $GW$ approximation four times, modifying only the poles of $G$ at every iteration.\cite{van2006quasiparticle} This process is compensating the lack of higher order correction of the G$_0$W$_0$ approximation and usually improves the agreement with experiments regarding the bandgap values.
\begin{figure}[h!b]
	\vspace{0.2cm}
	\setcapindent{2em}
	\centering
	\includegraphics[width=0.8\textwidth]{mBN_bands_LDA_G0W0_G4W4.pdf}
	\caption{Electronic bands of freestanding monolayer BN with different levels of theory : DFT (orange), G$_ 0$W$_0$ (green) and evG$_4$W$_4$ (red)} %MODIF : bigger labels and ticks and legends
	\label{fig:mBN_bands_LDA_G0W0_G4W4}
\end{figure}
This indirect gap is due to the presence of nearly-free electron states at $\Gamma$. 
In fact, the $GW$ self-energy does not correct the $\sigma^*$-like at $\Gamma$ as much as the $\pi$-like states around $K$ and $M$.
The nearly-free electron states have been investigated in BN nanotubes and mBN,\cite{blase1994stability,Blase1995monolayer} but only at the independent-particles, \acrshort{DFT} level. They may provide a possible mechanism for luminescence quenching.

Despite the presence of these states at $\Gamma$, the optical properties of BN-based systems are actually dictated by the $\pi$ bands around $K$ and $M$, and this remains true for mBN.
The optical spectrum of mBN is characterized by a strong doubly degenerate excitonic peak of symmetry $E$ at about $6$ eV. Exciton dispersions have been reported in several articles.\cite{cudazzo2016exciton,koskelo2017excitons,sponza2018direct} In Fig. \ref{fig:mBN_excdisp_lda_gw} we also report our calculated dispersion along selected high-symmetry lines, starting both from the quasiparticle band structure and the DFT one, where we used a so-called scissor operator to account for the lack of $GW$ correction so that the BSE starts with the same energy values.
\begin{figure}[h!b]
	\vspace{0.2cm}
	\setcapindent{2em}
	\centering
	\includegraphics[width=0.8\textwidth]{mBN_excdisp_lda_gw.pdf}
	\caption{Calculated exciton dispersion for monolayer hBN, starting from either the DFT-LDA eigenvalues with a scissor operator (blue) or the G$_0$W$_0$ quasiparticle energies (red). Dots represent our calculated BSE data, lines are Fourier interpolations}
    %MODIF math font for labels, thicker lines and markers
	\label{fig:mBN_excdisp_lda_gw}
\end{figure}
We found that excitons at momentum $\qq=K$ have a lower energy than the direct exciton at $\qq=0$ when starting from the $G_0W_0$, a feature inherited from the indirectness of the quasiparticle structure. In fact, these low-energy excitons are due to transitions towards the nearly-free electron states at $\Gamma$.
These new excitonic states are clearly distinguishable from the ``standard'' BN excitons by plotting their wavefunctions in real space, as it is done in the insets of Fig. \ref{fig:mBN_excdisp_wf} for several different center-of-mass momenta of the various states. We plot the electron distribution in real space when fixing the hole just above a Nitrogen atom, like it would be in a $p_z$ orbital. This is what we call the exciton wavefunction in real space.

We tracked the exciton wavefunction of the lowest two bands the high-symmetry lines. We can see that the exciton momentum confers the wavefunction a shape according to the symmetry of the point, \textit{e.g.} it has a straight shape at \MM~but is circular at $\Gamma$ and \KK. 
While the usual $\pi \rightarrow \pi^*$-derived states (green exciton ``bands'' in the figure) display the electronic density strongly localized on the Boron sublattice, when the hole is fixed on top of a Nitrogen, the $\pi \rightarrow \sigma^*$-derived states (red exciton ``band'') present an electron density strongly delocalized away from the layer plane. This is a clear signature of nearly-free electron character.
\begin{figure}[h!b]
	\vspace{0.2cm}
	\setcapindent{2em}
	\centering
	\includegraphics[width=0.95\textwidth]{1l_excdisp_disentangled.pdf}
	\caption{Details of the exciton dispersion of monolayer hexagonal BN. The insets show the spatial localization of the exciton wavefunction at several different $q$-points and branches (this is obtained by fixing the hole position on top of a Nitrogen atom, i.e. on a valence $\pi$ orbital, and plotting the resulting electron density). As evidenced in the insets, the red branch in the dispersion plot is due to the nearly-free electron states (involving conduction bands with $\sigma^*$ character), while the green branches originate from the optically active $\pi-\pi^*$ band transitions.}
	%MODIF : red arrows, bigger labels
	\label{fig:mBN_excdisp_wf}
\end{figure}

With this analysis and for other reasons that will be explained in Sec. \ref{sec:substrate}, we decided to use the \acrshort{DFT} eigenvalues with a scissor operator as a starting point of the \acrshort{BSE} and the subsequent steps, namely the calculation of exciton-phonon coupling and the luminescence spectrum.

%
\section{Exciton-phonon matrix elements resolved in momentum}
We can plot the calculated matrix elements over the \acrlong{BZ} thanks to $\qq$-dependence in Eq. \eqref{eq:Gkkp}. For the bulk hBN, we plot in Fig. \ref{fig:Gkkp_plot_hBN} the exciton-phonon coupling modulus for the lowest-lying finite-momentum excitons $\beta=1$ and $\beta=2$ scattered into the bright excitons $\lambda=3$ and $\lambda=4$ that are degenerate at $\Gamma$, for all phonon modes. We sum the excitons two by two because they are degenerate at $\Gamma$ and other points in the \acrshort{BZ}. We average over the $\qq_z$ points belonging to discrete planes orthogonal to the $\Gamma A$ line in order to have a two-dimensional plot. The quantity we plot is : 
\begin{equation}
    |\mathcal{G}_{3+4,1+2}(\qq_\parallel)| = \frac{1}{N_{q_z}}\left|\sum_{\mu,q_z} \mathcal{G}^\mu_{3+4,1+2}(q_z,\qq_\parallel)\right|
\end{equation}
We also plot the same quantity but keeping only ZA and ZO phonon modes in the sum. 
\begin{figure}[h!t]%
	\vspace{0.2cm}
	\setcapindent{2em}
	\centering
    \subfloat[All phonon modes.]{\label{Gkkp_plot_hBN:all_phonons} \includegraphics[width=0.45\textwidth]{hBN_all_phonons_all_planes.png}} \qquad 
    \subfloat[ZA+ZO modes only.]{\label{Gkkp_plot_hBN:z_only} \includegraphics[width=0.45\textwidth]{hBN_z_only_all_planes.png}}%
    \caption{Magnitude of the coupling between the finite-momentum excitons and the lowest-lying bright excitons in bulk hexagonal BN. Color bar is the modulus of $\mathcal{G}(\qq)$ in eV, for a 18$\times$18 $\qq$-points grid. \textcolor{blue}{add label for colorbar}}
	\label{fig:Gkkp_plot_hBN}
\end{figure}
It is the probability that the excitons $\beta=1$ and $\beta=2$ are scattered into the zero-momentum excitons $\lambda=3$ and $\lambda=4$ by all phonon modes with the corresponding momentum. The information we can extract from this plot is that the coupling has the same symmetry as the crystal, where the 6-fold rotation is clearly visible. From the panel \ref{Gkkp_plot_hBN:all_phonons}, we see that the scattering is maximal with excitons close to the $\Gamma A$ line. From the second panel \ref{Gkkp_plot_hBN:z_only}, we see that the ZA and ZO modes couple more with the minimum excitons on the $\Gamma K$ lines. This coupling contributes to about ten percent of the sum of all modes, as can be seen with the color bars.  

For the monolayer BN, we plot a similar quantity in Fig. \ref{fig:mBN_Gkkp}, except there is no need of averaging over planes since the \acrshort{BZ} is two-dimensional. We plot the scattering of finite-momentum excitons $\beta=1$ and $\beta=2$ into the two degenerate bright excitons $\lambda = 1$ and $\lambda = 2$ at $\Gamma$ (where the $\beta$ and $\lambda$ indices coincide).
\begin{figure}[h!t]
	\vspace{0.2cm}
	\setcapindent{2em}
	\centering
	\includegraphics[width=0.55\textwidth]{mBN_Gkkp.png}
	\caption{Magnitude of the coupling between the finite-momentum excitons and the lowest-lying bright excitons in monolayer hBN. Color bar is the modulus of $\mathcal{G}(\qq)$ in eV for a $\qq$-points grid of 36$\times$36 grid. \textcolor{blue}{add label for colorbar}} 
	\label{fig:mBN_Gkkp} %MODIF : trim + add label
\end{figure}
Here the situation is different since most of the coupling happens around $\Gamma$ and is about 4 times stronger than in the bulk materials. \textcolor{red}{Do we know why it is stronger in the monolayer ?} The 6-fold hexagonal pattern can still be distinguished with lower coupling strength. This result is a first hint that in \acrshort{mBN}, it is less likely to see phonon-satellites coming from the scattering of an exciton at the \acrshort{BZ} edge than at the center.   


%
\section{Luminescence spectra}
%
\subsection{Benchmark on bulk hBN} \label{sec:bulk_hBN}
We now put to the test our method by calculating the luminescence spectra of bulk hBN which will serve as a benchmark. Indeed we can compare it to our finite difference method as well as existing calculations in the literature and most importantly to experiments. We plot in Fig. \ref{fig:ph_exc_disp_hBN} the phonon and exciton dispersion of bulk \acrlong{hBN}. The phonon dispersion has the labels of the different modes.
\begin{figure}[h!t]%
	\vspace{0.2cm}
	\setcapindent{2em}
	\centering
    \subfloat[Phonon dispersion of hBN]{\includegraphics[width=0.42\textwidth]{hBN_phdisp.pdf}}
    \subfloat[Exciton dispersion of hBN \textcolor{blue}{ticklabel is wrong it is H instead of L}]{\includegraphics[width=0.48\textwidth]{hBN_excdisp.pdf}}
    \caption{Phonon (left) and exciton (right) dispersions of bulk hexagonal BN.}
    \label{fig:ph_exc_disp_hBN} %MODIF : trim and same size for both
\end{figure}
The exciton dispersion exhibits a double minimum on the $\Gamma K$ line with the Fourier interpolation. In fact the true minimum is on a point that is not on the $\Gamma K$ line. This is verified in Ref. \cite{zanfrognini2023distinguishing}.
With our coarse 18$\times$18$\times$6 momentum grid, the points with the minimum excitonic energies are labelled $R_1$ and $R_2$ and their reduced coordinates are fractions of the reciprocal lattice vectors $R_1=(\tfrac{1}{9},\tfrac{2}{9},0)$ and $R_2=(\tfrac{1}{9},\tfrac{5}{18},0)$. With the double-grid approach explained in Appendix \ref{app:comp_details_Chapt3}, the sampling of the exciton dispersion is much finer and the true minimum momentum is accurately located.

In the left panel of Fig. \ref{fig:hBN_PL_comparison}, we plot the luminescence spectra obtained with the exciton-phonon coupling from finite difference, as presented in Chapter 2, compared with the present \textit{ab initio} method. The peaks are given by a Dirac delta function with a finite broadening added to follow a Lorentzian shape and match the experimental peak shapes (more numerical details can be found in Appendix \ref{app:comp_details_Chapt3}). The shape of the LA/TA phonon satellites on the high energy side of the spectrum, computed with the present \textit{ab initio} method, are broader than the single Lorentzian peaks of the finite difference method. This is a consequence of the integration on the full $\qq$-grid present in the former method and not in the latter. In addition, we used a double-grid for the exciton energies, the phonon frequencies and the electron-phonon coupling matrix elements, so that the numerical instabilities of the renormalization factor in Eq. \eqref{eq:renorm_fact} are smoothed out and the dispersions are accurately described. We verified that we obtain similar spectra when we restrict the sum on $\qq$ in Eq. \eqref{eq:I_PL} to the $\bar{q}$ points used in Chapter 2. The difference comes from the renormalization due to the denominators in the self-energy Eq. \eqref{eq:excph_SE_compact} which is missing in the finite difference formula. It should also be noted that the inclusion of phonon absorption processes does not give additional peaks in the spectrum. Indeed, the satellites due to phonon absorption have an intensity proportional to the Bose-Einstein occupation of phonons, which is low for the lattice temperature of 6 K we simulated. We have verified that these peaks appear when the lattice temperature is increased. Similarly, we have verified that higher-energy excitons become populated by the Boltzmann occupation function when we increased $T_{exc}$ and produce satellite peaks in the spectrum. 
\begin{figure}[h!b]%
	\vspace{0.2cm}
	\setcapindent{2em}
	\centering
    \subfloat[Comparison of finite-difference and \textit{ab initio} methods.]{\includegraphics[width=0.45\textwidth]{comparison_pl.png}} \label{comparison_fdd} \qquad 
    \subfloat[Comparison of our \textit{ab initio} method and the one of Chen \textit{et al.}]{\label{comparison_ai} \includegraphics[width=0.45\textwidth]{hbn_pl_bulk.pdf}}%
    \caption{Comparisons of the normalized luminescence spectrum obtained with our \textit{ab initio} method (blue line) and the finite difference method (green line) on the left panel. On the right, we compare it to the result of Ref. \cite{chen2020exciton} (red dashed line). In both panels, the experimental data (black dots) comes from Ref. \cite{schue2019bright}.} %MODIF : trim and same size for both
	\label{fig:hBN_PL_comparison}
\end{figure}


In the right panel of Fig. \ref{fig:hBN_PL_comparison}, we also compare our result with the spectrum obtained by Chen \textit{et al.} in Ref. \cite{chen2020exciton}. As mentioned previously, the exciton-phonon matrix elements we compute are the same than in their formulation, if we do the correct change of variable to account for their different momentum conservation convention. We implemented their convention in \yambo~and verified that the spectra do not change when using one or the other. They compute the luminescence intensity differently than the van Roosbroeck--Shockley relation, this is why the spectra look different. Our spectrum reproduces correctly the position of the satellites measured in Ref. \cite{schue2019bright} (note that all spectra have been shifted to match the energy of the experimental peaks) and the intensity of the LA/TA doublet on the high energy side, which is an improvement compared to the results of Chen \textit{et al.}. However, the intensity of the LO/TO doublet on the low energy side is not well reproduced. It is in fact inverted, with the TO peak being less intense than the LO one. Since this inaccuracy in the intensity if still in the correct order of magnitude, we decided to proceed with this implementation. It has been shown very recently in Ref. \cite{zanfrognini2023distinguishing} that this effect can be corrected by solving the \acrshort{BSE} at $\Gamma$ without the long-range component of the Coulomb interaction, hence using $\bar{\chi}$ from Eq. \eqref{eq:chi_bar} instead of the full $\chi$ from Eq. \eqref{eq:chi_iL}. We can also notice that the experimental peaks have phonon overtones due to the scattering with multiple phonon which give them an asymmetric shape,\cite{vuong2017exciton} and this is not included in our framework.

Besides, another issue in the spectrum is the presence of a low intensity peak at 5.93 eV, which appear neither in other numerical spectra, nor in experimental measurements. To investigate the origin of this peak, we can separate the contribution of each phonon mode to the spectrum. We plot said contributions in Fig. \ref{fig:hBN_split_phonons}
\begin{figure}[h!t]%
	\vspace{0.1cm}
	\setcapindent{2em}
	\centering
    \includegraphics[width=0.6\textwidth]{hbn_pl_split_phonons.pdf}
    \caption{Luminescence spectrum of bulk hBN resolved with respect to the different
    phonon modes on the left panel and phonon dispersion on the right panel.}
	\label{fig:hBN_split_phonons}
\end{figure}
We see that the additional peaks come from scattering of excitons with ZO and ZA phonon modes. In our simulations, we set the light polarization to be in-plane. Hence optically created excitons are in-plane, and their scattering with out-of-plane phonons (namely the ZA and ZO modes) and their successive recombination is forbidden by symmetry.\cite{paleari2019exciton,cassabois2016hexagonal} If the crystal symmetries are changed, then these forbidden peaks can appear in photoluminescence. It is the case for rhombohedral BN.\cite{zanfrognini2023distinguishing} In our case, the problem arises from the definition of the exciton-phonon matrix elements in Eq. \eqref{eq:Gkkp}. The electron-phonon matrix elements and the exciton eigenvectors have different random phases that depend on the different sets of Kohn-Sham wavefunctions that were used to generate them in the first place. It is a non-trivial technical and numerical issue to account for these phases consistently.  Indeed, some \acrshort{DFPT} implementations (like \textsc{Quantum ESPRESSO}) recalculate the KS wavefunctions at $\kk+\qq$ for each $\qq$. Instead, a single set of wavefunctions is used to define the BSE matrix at any momentum $\QQ$ in the \yambo code, where the $\kk+\qq$ wavefunctions are obtained by symmetry transformations, thus imposing a specific choice of the relative phase between the wavefunctions. This difference causes a phase mismatch in the definition of the exciton-phonon matrix elements, Eq.~\eqref{eq:Gkkp}, because both the electron-phonon matrix element and the excitonic coefficients enter as full complex numbers. This is likely the reason why the magnitude of the coupling with ZA and ZO phonon modes is as large as displayed in Fig. \ref{fig:hBN_PL_comparison} (b).
This issue remains also if the electron-phonon matrix elements are obtained via Wannier interpolation\cite{chen2020exciton}, since the wavefunction used to construct the excitonic matrix would be different from the ones resulting via the Wannier procedure. In this case the interpolation process should be modified by fixing the wavefunction phases\cite{giustino2007electron}. 
The phase mismatch is not present in calculations based on finite differences\cite{paleari2018excitons,lechifflart2022excitons} because in this case exciton-phonon coupling is directly calculated as a derivative of the exciton dipole matrix elements on a supercell. However, these types of calculations are restricted to a single $\qq$-vector.
In the case of hBN luminescence, we verified that the phase mismatch only gives small changes in the numerical results (by testing different sets of wavefunctions with different random phases). A possible workaround was investigated very recently in Ref. \cite{zanfrognini2023distinguishing} but it requires to write an interface with a third simulation code and most importantly to turn off all crystal symmetries, which considerably slows down all calculation and requires a lot more disk space.

Overall, our spectrum is in fairly good agreement with the experimental one. Keeping in mind the issues discussed above, we turn to the study of a case where the main advantage of our method is fully exploited : the fact that we can compare the relative intensities of direct peaks and phonon satellites.

%
\subsection{Results on mBN}
\begin{figure}[H]
	\vspace{0.2cm}
	\setcapindent{2em}
	\centering
	\includegraphics[width=0.9\textwidth]{hBN_monolayer_lum2.pdf}
	\caption{Calculated luminescence spectrum of monolayer hBN (a) compared to the experimental results of Ref. \cite{elias2019direct}(b), Ref. \cite{rousseau2021monolayer}(c) and Ref. \cite{wang2022scalable}(d). The theoretical spectrum has been shifted to match the main experimental peak (c). For clarity, we have plotted the theoretical spectrum next to each experimental result. In the inset of panel (a) we show the theoretical spectrum in a logarithmic scale, revealing the presence of a small phonon satellite.} 
	\label{fig:mBN_PL} %MODIF: put the scale for the log inset
\end{figure}
We plot the central result of this chapter in Fig. \ref{fig:mBN_PL}. In panel (a) we report our luminescence calculations of a single layer m-hBN compared to the measurements of Refs. \cite{elias2019direct,wang2022scalable,rousseau2021monolayer}, panels (b),(c),(d). Beside the main direct emission peak, we found a satellite at lower energy that has a small intensity, about two orders of magnitude lower than the direct peak (see inset in logarithmic scale in the panel (a)). We were able to identify the different terms contributing to the satellite intensity thanks to the internal sum in Eq. \eqref{eq:I_PL}, both in terms of phonons modes and in momentum. Thus we identify the satellite as a scattering from a zero-momentum exciton due to the LO and TO phonons.

Therefore, the additional peaks seen in some of the experiments are not explained by our calculations. We also included possible indirect transition from excitons with momentum corresponding to the $K$ point due to the $\pi \rightarrow \pi^*$ transitions (relative minimum of the green curve in Fig. \ref{fig:mBN_excdisp_wf}). However, we found that due to the relatively large energy difference of $0.14$~eV between direct and indirect excitons, the contribution of these latter states to the luminescence spectrum is null.\footnote{We did not consider polaritonic effects that could slightly modify the luminescence spectra, see Ref. \cite{henriques2019optical} for a discussion.}

As a sanity check, we considered the possibility that the distance between the exciton $\KK$ and $\Gamma$ is not well reproduced by our calculations and analyzed the matrix elements of the phonon-assisted transitions between $\KK$ and $\Gamma$. We found that these are in any case too small to explain the additional peak seen close to the main one in the experiments with a Graphite substrate.

In the light of these results, let us now discuss the different experiments. The details of the three luminescence spectra reported in the literature, see Fig.~\ref{fig:mBN_PL}(b-d), are the following: two of them feature mBN grown by molecular beam epitaxy on Graphite,\cite{elias2019direct,wang2022scalable} and one mechanically exfoliated on Silicon Oxide.\cite{rousseau2021monolayer}
With the Graphite substrate, multiple peaks are visible.
These peaks have been attributed to various causes, which we will briefly summarized here. In the work of Elias \emph{et al.},\cite{elias2019direct} the possibility of additional satellites appearing due to scattering of the indirect excitons at $K$ was considered. We also mention that in a very recent article, not peer-reviewed yet,\cite{shima2023cathodoluminescence} cathodoluminescence measurements of mBN on Graphite revealed a faint peak at 6.04 eV, that was attributed to the phonon-assisted recombination of the $K$ exciton assisted with a ZA phonon.
In the work of Rousseau \emph{et al.}\cite{rousseau2021monolayer} they put forward the possible presence of bubbles in the sample as cause of the additional peaks. 
Finally, in the article of Wang \emph{et al.}\cite{wang2022scalable}, these additional peaks were attributed to the presence of multilayer BN regions and the peak at the middle of the three is caused by defects that would allow the triplet excitons (out of the scope of this thesis) to recombine radiatively. 


Our theoretical work allows us to rule out the first hypothesis since, as shown above, the energy difference between $\Gamma$ and $\KK$ is large and the phonon-assisted transitions have too low an intensity to have indirect excitons visible in luminescence in the energy range where the second experimental peak appears, while the $\pi\rightarrow\sigma^*$ transitions seem to play no role. Regarding the effect of bubbles in the sample on the luminescence spectra, we have shown in the previous Chapter that strain can induce shifts of the luminescence spectra.\cite{lechifflart2022excitons}
Yet, in order to obtain a significant effect, the strain must be very large, and in addition it is difficult to explain with strain the presence of two well-defined peaks, like those visible in the experimental spectra. 
Finally, there is the hypothesis of defects or multi-layers BN. We think this is the most plausible hypothesis, because it has been shown that some defects can produce levels close to the main exciton,\cite{attaccalite2011coupling} and multi-layers BN induce splittings of the main peak.\cite{paleari2018excitons} 
Finally, note that the presence of defects or edges, which break translational symmetry, could make the indirect exciton visible without the need for phonon mediation.\cite{feierabend2017proposal}


%
\section{Effects of the substrate on the electronic bands and excitons of mBN} \label{sec:substrate}
In order to see if we can really expect an optical experimental signature from the exciton made of nearly-free electron states -- that will make the system functionally indirect -- we decided to investigate how the presence of a substrate modifies their position with respect to the direct gap, compared to the freestanding mBN. We included a graphitic substrate in the simulation, analogous to the one used in some of the experiments, and found that it lowers the direct gap at $K$ much more than the indirect one, actually making the system a true direct band gap insulator again. This is most likely due to the stronger interaction of the $p_z$ orbitals of boron with those of Graphene while the planar $\sigma^*$ states are less affected. The dependence of the mBN gaps on the number of substrate Graphene layers is shown in Fig. \ref{fig:mBN_gap_layers} (calculation details are given in \textcolor{red}{I can't find the details.}%the Appendix \ref{app:comp_details_Chapt3}). 
\begin{figure}[H]
	\vspace{0.2cm}
	\setcapindent{2em}
	\centering
	\includegraphics[width=0.8\textwidth]{mBN_gap_vs_layers.pdf}
	\caption{Band gaps of mBN as a function of the number of graphene layers\label{gap_vs_layers}. The large direct gap at $\Gamma$ is in blue, the indirect $\pi\rightarrow\sigma^*$, i.e. $K\rightarrow\Gamma$, is in green and the smallest $K\rightarrow K$ direct gap is in red. \textcolor{red}{Is it the GW gap ? the value seems pretty high}} %MODIF : légendes plus grosses, lignes et points plus gros (check my paper with Conor for inspiration) 
	\label{fig:mBN_gap_layers}
\end{figure}
Therefore, we expect that these states at $\Gamma$ will not contribute to the luminescence in a realistic experiment where mBN is deposited or grown on a substrate.
In order to simulate luminescence from an ideal mBN deposited on a substrate we started from the LDA band structure and applied a scissor operator that allows us to maintain the direct nature of mBN (\textit{i.e.} removing the red ``band'' in Fig. \ref{fig:mBN_excdisp_wf}). 

We investigated further the effect of the increased screening due to the substrate by applying a distortion to the exciton dispersion. Indeed, with increased screening the attractive term in the BSE kernel is stronger, hence increasing the finite-momentum excitons binding energy. By adding a parameter that renormalizes the exciton energies proportionally to their momentum, we can roughly simulate this effect and bring the exciton at \KK~lower in energy than the one at $\Gamma$. Hence, this new minimum gets more populated with the Boltzmann occupation function and it might increase the intensity of phonon satellites. An \textit{ab initio} treatment of the Graphite substrate is possible at the $GW$ and BSE level with recent numerical developments,\cite{guandalini2023efficient} but it remains too expansive computationally for our way of calculating the exciton-phonon coupling, given the number of $\kk$ and $\qq$ points needed to accurately describe the sharp electronic dispersion of Graphene layers. More advanced numerical procedures to include the screening of a substrate or any external environment exist,\cite{ugeda2014giant,bradley2015probing} but they are beyond the scope of this thesis.
\begin{figure}[H]
	\vspace{0.2cm}
	\setcapindent{2em}
	\centering
	\includegraphics[width=0.8\textwidth]{mBN_distortion.pdf}
	\caption{Left column : plots of the Fourier interpolated lowest excitonic band in the dispersion of mBN, between $\Gamma$ and \KK, when momentum-dependent distortion is applied to decrease the energy at \KK. The Boltzmann occupation function is represented in green. Right column : corresponding luminescence plots, normalized to 1. The main peak is always visible and the phonon satellite gain intensity when the distortion brings the energy at \KK~lower than at $\Gamma$. \textcolor{blue}{I will put arrows or dashes to symbolize the distortion}} %MODIF : put math font for labels, think about colors; add arrow to symbolize distortion: more vspace on xlabel left to highlight Gamma and K; trim
    \label{fig:mBN_distortion}
\end{figure}
We plot the undistorted exciton dispersion and luminescence spectrum in the two upper panels of Fig. \ref{fig:mBN_distortion}. The phonon satellite coming from the \KK~valley is invisible due to the energy separation. We see that when $\Delta_{K\Gamma} \equiv E_1(\KK) - E_1(\Gamma) = 1 meV$, the valley at \KK~gets populated, but the corresponding luminescence plots still show a phonon satellite barely visible compared to the direct peak. It is an overlap of the satellites from $\Gamma$ and from \KK, that scatter with the same phonon modes and hence have the same energy here. Finally, when $\Delta_{K\Gamma} = -27$ meV, the \KK~valley is much more populated than the $\Gamma$ one and thus the phonon satellite becomes visible. However, we left the exciton-phonon matrix elements unchanged for this specific plot, which is probably a bad approximation since the Graphite substrate screening has an effect on the electron-phonon coupling.\cite{sohier2021remote} It means that the change in the luminescence spectra with the distortion is only due to the change in exciton energies at finite-momentum and to the Boltzmann occupation function. As a consequence, the satellite in the bottom-right panel of Fig. \ref{fig:mBN_distortion} appears because the direct peak is less intense. In any case, to see this effect the \KK~valley needs to be renormalized of about 170 meV, which seems unrealistic for a Graphite substrate. \textcolor{red}{Is it unrealistic ?} This is another indication to rule out the possibility that the satellite coming from \KK~is as bright as the direct peak in the experiments.

%
\section{Preliminary results on bBN}
The Bernal form as a similar crystal structure as \acrshort{hBN}, but the stacking of the layers is AB instead of AA', with a Boron atom lying above the center of an hexagon.
For this material, we started with the same computational parameters as the more studied bulk \acrshort{hBN}. This is the start of a detailed study of how different functionals at the \acrshort{DFT} level and different convergence thresholds can change the optimal lattice parameter, the phonon and exciton dispersion, thus the exciton-phonon coupling and the optical spectra. 
The Bernal stacking type exhibits quite a different exciton dispersion compared to hBN, as displayed in Fig. \ref{fig:bBN_excdisp}. The relative minimum along $\Gamma$-K now has an higher energy than the one at $\Gamma$. The energy of the exciton at $\Gamma$ is 5.39 eV and the minimum at $\qq=(\tfrac{1}{6},\tfrac{1}{6},0)$, which is $|\Gamma K|/2$, is 5.41 eV. This $20$ meV difference means that the indirect exciton will be populated at large effective temperatures by the Boltzmann occupation function, but only sparsely in a low-temperature measurement. We then expect most of the photoluminescence intensity coming from the direct exciton, with very little phonon-assisted satellite peaks in the spectrum.
\begin{figure}[H]
	\vspace{0.2cm}
	\setcapindent{2em}
	\centering
	\includegraphics[width=0.8\textwidth]{bBN_excdisp.pdf}
	\caption{Calculated exciton dispersion in Bernal BN \textcolor{blue}{ticklabel is wrong it is H instead of L}} %	
    \label{fig:bBN_excdisp}
\end{figure}

We plot the exciton-phonon matrix elements for \acrshort{bBN} in Fig. \ref{fig:Gkkp_plot_hBN}. The coupling for all phonon modes is very similar than for \acrshort{hBN}, with a 6-fold snowflake shape. The maximum of the coupling is around $\Gamma$ with some coupling along the $\Gamma K$ lines. Concerning the ZA and ZO modes, the matrix elements are more homogeneous with maxima at the middle of the $\Gamma K$ lines and the coupling is non-zero over the whole \acrshort{BZ}, although the magnitude is slightly lower than for hBN.
\begin{figure}[h!b]%
	\vspace{0.2cm}
	\setcapindent{2em}
	\centering
    \subfloat[All phonon modes.]{\label{Gkkp_plot_bBN:all_phonons} \includegraphics[width=0.45\textwidth]{bBN_Gkkp_allphonons.pdf}} \qquad 
    \subfloat[ZA+ZO modes only.]{\label{Gkkp_plot_bBN:z_only} \includegraphics[width=0.45\textwidth]{bBN_Gkkp_zonly.pdf}}%
    \caption{Magnitude of the coupling between the finite-momentum excitons and the lowest-lying bright excitons in Bernal BN. Color bar is the modulus of $\mathcal{G}(\qq_\parallel)$ in eV, for a 18$\times$18 $\qq$-points grid. \textcolor{blue}{add label for colorbar}}
	\label{fig:Gkkp_plot_bBN}
\end{figure}

We found most of the contribution to the luminescence came from excitonic states at $\Gamma$ and along the $\Gamma A$ line which has a low dispersion (see Fig. \ref{fig:bBN_excdisp}). This is in agreement with the exciton-phonon matrix elements showing a higher coupling around $\Gamma$.
\begin{figure}[h!b]
	\vspace{0.2cm}
	\setcapindent{2em}
	\centering
	\includegraphics[width=0.8\textwidth]{bbn_pl.pdf}
	\caption{Plots of the calculated luminescence spectrum of Bernal BN, with and without exciton energy distortion, compared to experiments. We shifted the calculated spectra to match the experimental peaks.} %MODIF : align and normalize the peaks better
    \label{fig:bBN_PL}
\end{figure}
However with the distortion of the excitonic dispersion, some satellite peaks appear with intensities comparable to the direct peak. The luminescence plot with separated phonon contributions in Fig. \ref{fig:bBN_PL_split_phonons} allows to identify the peaks that are absent in the experiments and that are made visible with the distortion. Here we plot only the satellite contributions to luminescence. These satellites come from scattering with excitons in the nex minimal valley on the $\Gamma K$ line. We see that one of the major contribution comes from a ZO phonon mode, which should be forbidden by symmetry. This is the same numerical issue that was explained above, for the ZA and ZO peaks in the spectra of \acrshort{hBN}.
\begin{figure}[h!b]
	\vspace{0.2cm}
	\setcapindent{2em}
	\centering
	\includegraphics[width=0.8\textwidth]{bbn_pl_split_phonons.pdf}
	\caption{Plots of the satellite contributions to the luminescence spectrum of Bernal BN, where the different phonon mode contributions are separated (shifted to match the experimental peaks).} %MODIF : same modif as for hBN; shift like experiment
    \label{fig:bBN_PL_split_phonons}
\end{figure}

Since a slight change in the exciton energies can change drastically the luminescence spectrum, we need to perform a careful study of all the numerical parameters involved in the workflow. This is still ongoing work.


\section*{Conclusion of the chapter}

In this Chapter, we presented a first-principles methodology to calculate phonon-assisted luminescence in exciton-dominated materials. It is based on a dynamical correction to the static Bethe-Salpeter equation given by an excitonic self-energy term describing exciton-phonon interaction. Using this self-energy, we obtained a formula for the optical response that contains corrections up to first order in the exciton-phonon coupling.
Unlike previous formulations, we are also able to calculate the renormalization factor for direct transitions, which allows for a quantitative comparison between direct and phonon-assisted emission signatures.
From the optical response function, and employing a steady-state approximation, we obtained a formula for the phonon-assisted luminescence. All ingredients that enter in this formulation have been calculated \emph{ab initio}, except for the excitonic temperature relative to the occupation of excitonic states.
We first validate our approach on bulk \hbn, where clear and well-established experiments exist. We then applied this approach to the BN single layer, where recent discordant photoluminescence measurements were reported independently by different groups. In mBN we found that the luminescence spectrum is dominated by the single direct peak only and phonon replicas, while present, have negligible intensity. In addition, 
phonon-assisted transitions from the lowest indirect exciton remain too low in intensity to explain the measured spectra. Therefore, we rule out phonon-assisted processes as the cause of the additional spectral fine structure sometimes seen in experiment.
We support the interpretation that this fine structure is not intrinsic, nor due only to substrate effects, but depends on sample quality.

The early conclusions of the recently started numerical study on Bernal BN is that taking the experimental value for the lattice parameters and an \acrshort{LDA} functional give electron and phonon dispersions that are very close from those obtained with more advanced functionals including interlayer van der Waals interactions. 

Finally, we mention that our methodology based on the dynamical self-energy is fully implemented in the \yambo code and applicable to other systems of interest. This formulation allows one to obtain more observables than just the luminescence presented here, such as phonon-assisted absorption, exciton linewidths and relaxation rates.
