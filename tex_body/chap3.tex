\chapter{Ab initio exciton-phonon coupling}
\chaptertoc{}

\section{Introduction}
Mainly on monolayer, the three experiments
mention also the cathodoluminescence paper in which we see only the direct peak
say we did a benchmark on hBN
talk about the bBN and the reason we chose it : direct and indirect gaps are close in energy so we might see direct peak and phonon satellites. Indeed it was see experimentally so it is a good candidate for our method.\\
Chen Sangalli Bernardi.
cite Pedro Melo and say there is a different theoretical framework out of equilibrium, no need of steady-state. However, including the phonon-assisted transitions there is very challenging.\\
thus calculating the renormalization of the direct peak induced by the indirect transitions in the luminescence spectra.

\section{monolayer exciton dispersion}
fitting of the dispersion, Lbar vs Lfull
the nearly-free electron states at Gamma

\section{Theory of the ab initio exciton-phonon coupling}
In this section we present an \textit{ab initio} approach to obtain the exciton-phonon coupling matrix elements, that goes beyond the finite-difference approach presented in Sec. \ref{sec:excph_fdd}. The great advantage of this formulation is the possibility to integrate over exciton momenta in the full \acrlong{BZ} Brillouin Zone and to obtain the coupling of all excitons to all phonons.\\
For this work we based our numerical implementation on the formula derived by our collaborator Fulvio Paleari in his PhD thesis, \cite{paleari2019first} itself stemming from the theoretical work of Pierluigi Cudazzo published in Ref. \cite{cudazzo2020first}. The electron-phonon interaction induces a perturbation of the electron-hole interaction present in the BSE kernel. Hence, a dynamical correction appears in the kernel, which allows to construct a self-energy for the exciton-exciton interaction mediated by phonons. I refer the interested reader to these two references, where the rigorous derivation can be found.\\

In this thesis, I will use first-order perturbation theory for the excitonic Hamiltonian, which introduces an interaction term due to the electron-phonon coupling. 
\textcolor{red}{c'est nul faut que je reformule de façon cohérente : relis l'article}

We treat the excitons as boson-like particles bound by the Coulomb interaction so we can define a bosonic Hamiltonian for excitons :
\begin{equation}
    H_{\text{exc}} = \sum_\lambda E_\lambda (\QQ) \hat{a}^\dagger_{\lambda\QQ} \hat{a}_{\lambda\QQ}
\end{equation}
where $\hat{a}^\dagger_{\lambda\QQ}, \hat{a}_{\lambda\QQ}$ are the creation/annihilation operators for an exciton $\lambda$ with center-of-mass momentum $\QQ$ and energy $E_\lambda(\QQ)$. This is an approximation that ignores the fact that excitons are a pair of two bound fermions, this is why it works best at low exciton density so that the exciton are weakly interacting. It has been shown to correctly reproduce several experimental results.\cite{paleari2019exciton,perebeinos2005effect} However, there are theoretical evidence that the fermionic character of excitons cannot always be neglected.\cite{katzer2023excitonphononscattering}\\
There is also a more general approach where the idea is to treat the electron-electron, the electron-phonon interactions and the external field on the same footing,\cite{paleari2022exciton} which lifts some of the approximation we use, but it does not introduce relevant changes for the systems investigated here.\\





We consider a system with displacements from the equilibrium positions $\boldsymbol{u}_{Ls}$ ($L$ labels the unit cell and $s$ the atom). Once again, we start from the \acrshort{DFT} level and take the Taylor expansion of the Kohn-Sham potential, which was defined as $v_{\text{eff}}$ in Eq. \eqref{eq:KS_potential} and that we call here $V^{KS}$. The expansion around the equilibrium positions reads :
\begin{equation}
    V^{KS}(\left\{ \uu_{Ls} \right\}) = V_0^{KS} + \sum_{Ls\alpha} \frac{\partial V^{KS}}{\partial \uu_{Ls\alpha}} \uu_{Ls\alpha} + \mathcal{O}(\left\{ \uu_{Ls} \right\}^2)
\end{equation}
The electronic wave functions and eigenvalues of the perturbed system depend on the atomic displacements $\left\{ \uu_{Ls} \right\}$. To obtain their change in the perturbed system, we apply first-order perturbation theory by keeping terms linear in $\left\{ \uu_{Ls} \right\}$. To first order, the correction to the eigenvalues vanishes while the correction to the Kohn-Sham wave functions $\psi_i$, (solutions of Eq. \eqref{eq:KS_eqs}) can be written as : 
\begin{equation}
    \delta\ket{\psi_i} = \sum_{j\neq i} \frac{\bra{\psi_j} \Delta V \ket{\psi_i}}{\epsilon_i - \epsilon_j} \ket{\psi_j}, \qquad \text{with} \ \ \Delta V = \sum_{Ls\alpha}\frac{\partial V^{KS}}{\partial \uu_{Ls}}\cdot \uu_{Ls}
\end{equation}
In the following, we use the tilde for quantities of the perturbed system and write the perturbed wave function as :
\begin{equation}
    \ket{\tilde{\psi}_i} = \ket{\psi_i} + \delta\ket{\psi_i} = \ket{\psi_i} + \sum_{j \neq i} \Delta_{ij} \ket{\psi_j} \label{eq:perturb_wf}
\end{equation}
with
\begin{equation}
    \Delta_{ij} \equiv \frac{\bra{\psi_j} \Delta V \ket{\psi_i}}{\epsilon_i - \epsilon_j} \label{eq:Delta_var}
\end{equation}

We set ourselves in the Tamm-Dancoff approximation and we use the resonant Hamiltonian from Eq. \eqref{eq:H_BSE_res} as the  Hamiltonian of the unperturbed system $H \equiv H^{2p}(\{\boldsymbol{u}_{Ls} \}=0)$. For the perturbed system, we have $\Tilde{H} \equiv H^{2p}(\{\boldsymbol{u}_{Ls} \})$. 
The perturbed Hamiltonian matrix element is :
\begin{equation}
    \Tilde{H}_{\Tilde{v}\Tilde{c},\Tilde{v}'\Tilde{c}'}  = \bra{\Tilde{v}\Tilde{c}} \tilde{H} \ket{\Tilde{v}'\Tilde{c}'}  = ( \Tilde{\epsilon}_c - \Tilde{\epsilon}_v ) \delta_{\Tilde{v}\Tilde{v}'}\delta_{\Tilde{c}\Tilde{c}'} + \tilde{\Xi}_{\tv\tc}^{\tv'\tc'}
\end{equation}
where $v,c$ refer to valence and conduction bands, respectively.
The perturbed Bethe-Salpeter kernel is defined just as in Eq. \eqref{eq:BSE_kernel_vc}, except that it is evaluated with the screened interaction of the perturbed system $\tilde{W}$, and its matrix elements are expressed in the perturbed basis.

Solving the \acrshort{BSE} in Eq. \eqref{eq:BSE_secular} gives the exciton wave functions that we will name $\ket{\lambda}$ and energies $E_{\lambda}$ :
\begin{align}
    \sum_{v',c'} H_{vc,v'c'} A_{\lambda}^{v'c'} &= E_{\lambda} A_{\lambda}^{vc} \\
    \ket{\lambda} = \sum_{vc}& A^{vc}_{\lambda}\ket{vc}
\end{align}
To derive the exciton-phonon interaction, we project the perturbed BSE Hamiltonian onto the unperturbed basis set and keep only the terms to first-order in the phonon perturbation. 
By such a process, the terms that will arise and be different from the unperturbed BSE Hamiltonian will define the exciton-phonon interaction.
One can show that to first order, the perturbed and unperturbed electronic energies coincide, so we will use $\Tilde{\epsilon}_i = \epsilon_i$. 
The perturbed Hamiltonian in the unperturbed basis is :
\begin{align*}
    \Tilde{H}_{\lambda\lambda'} = \bra{\lambda'} \Tilde{H} \ket{\lambda} &= \sum_{\Tilde{v}\Tilde{c},\Tilde{v}'\Tilde{c}'}  \bra{\lambda'}\ket{\Tilde{v}\Tilde{c}}\bra{\Tilde{v}\Tilde{c}} \Tilde{H} \ket{\Tilde{v}'\Tilde{c}'}\bra{\Tilde{v}'\Tilde{c}'}\ket{\lambda} \\
    &= \sum_{vc,v'c'}\sum_{\Tilde{v}\Tilde{c},\Tilde{v}'\Tilde{c}'} \bra{\lambda'}\ket{vc}\bra{vc}\ket{\Tilde{v}\Tilde{c}}\bra{\Tilde{v}\Tilde{c}} \Tilde{H} \ket{\Tilde{v}'\Tilde{c}'}\bra{\Tilde{v}'\Tilde{c}'}\ket{v'c'}\bra{v'c'}\ket{\lambda}
\end{align*}
where we used the completeness relations of both basis sets, $\sum_{vc}\ket{vc}\bra{vc} = \mathds{1}$ and $\sum_{\Tilde{v},\Tilde{c}} \ket{\Tilde{v}\Tilde{c}}\bra{\Tilde{v}\Tilde{c}} = \mathds{1}$. By definition of the BSE wave functions $\bra{v'c'}\ket{\lambda} = A_{\lambda}^{v'c'}$, then we can write the above equation as :
\begin{equation}
    \Tilde{H}_{\lambda\lambda'} = \bra{\lambda'} \Tilde{H} \ket{\lambda} = \sum_{vc,v'c'} A_{\lambda'}^{vc*} A_{\lambda}^{v'c'} \times \left[ \sum_{\Tilde{v}\Tilde{c},\Tilde{v}'\Tilde{c}'} \bra{vc}\ket{\Tilde{v}\Tilde{c}}  \bra{\Tilde{v}\Tilde{c}} \Tilde{H} \ket{\Tilde{v}'\Tilde{c}'} \bra{\Tilde{v}'\Tilde{c}'}\ket{v'c'} \right]
    \label{eq:exc_ham}
\end{equation}
The term inside the square brackets can be separated in two :
\begin{align*}
    &\sum_{\Tilde{v}\Tilde{c},\Tilde{v}'\Tilde{c}'} \bra{vc}\ket{\Tilde{v}\Tilde{c}}  \bra{\Tilde{v}\Tilde{c}} \Tilde{H} \ket{\Tilde{v}'\Tilde{c}'} \bra{\Tilde{v}'\Tilde{c}'}\ket{v'c'} =  \sum_{\Tilde{v}\Tilde{c},\Tilde{v}'\Tilde{c}'} \bra{vc}\ket{\Tilde{v}\Tilde{c}}  \left[ (\Tilde{\epsilon}_{\Tilde{c}} - \Tilde{\epsilon}_{\Tilde{v}}) \delta_{\Tilde{v}\Tilde{v}'}\delta_{\Tilde{c}\Tilde{c}'} + \tilde{\Xi}_{\tv\tc}^{\tv'\tc'} \right] \bra{\Tilde{v}'\Tilde{c}'}\ket{v'c'} \\
    &= \sum_{\Tilde{v}\Tilde{c}} \bra{vc}\ket{\Tilde{v}\Tilde{c}} (\epsilon_{\Tilde{c}} - \epsilon_{\Tilde{v}}) \bra{\Tilde{v}\Tilde{c}}\ket{v'c'} + \sum_{\Tilde{v}\Tilde{c},\Tilde{v}'\Tilde{c}'} \bra{vc}\ket{\Tilde{v}\Tilde{c}} \tilde{\Xi}_{\tv\tc}^{\tv'\tc'} \bra{\Tilde{v}'\Tilde{c}'}\ket{v'c'}
\end{align*}
We make the choice to approximate the perturbed kernel with the unperturbed one, $\tilde{\Xi}_{\tv\tc}^{\tv'\tc'} \approx \bra{\Tilde{v}\Tilde{c}} \Xi \ket{\Tilde{v}'\Tilde{c}'}$, that is to say the effect of the atomic displacements screened interaction can be neglected and $W \approx \Tilde{W}$. This is the same approximation we took in Chapter 2 when we evaluated the response function by finite difference derivative. With this approximation we have :
\begin{equation}
    \sum_{\Tilde{v}\Tilde{c},\Tilde{v}'\Tilde{c}'} \bra{vc}\ket{\Tilde{v}\Tilde{c}} \tilde{\Xi}_{\tv\tc}^{\tv'\tc'} \bra{\Tilde{v}'\Tilde{c}'}\ket{v'c'} 
    \approx  \sum_{\Tilde{v}\Tilde{c},\Tilde{v}'\Tilde{c}'} \bra{vc}\ket{\Tilde{v}\Tilde{c}} 
    \bra{\Tilde{v}\Tilde{c}} \Xi \ket{\Tilde{v}'\Tilde{c}'}\bra{\Tilde{v}'\Tilde{c}'}\ket{v'c'} = \bra{vc} \Xi \ket{v'c'} = \Xi_{vc}^{v'c'}
\end{equation}
and thus the term in square brackets in Eq. \eqref{eq:exc_ham} becomes 
\begin{equation}
    \sum_{\Tilde{v}\Tilde{c},\Tilde{v}'\Tilde{c}'} \bra{vc}\ket{\Tilde{v}\Tilde{c}}  \bra{\Tilde{v}\Tilde{c}} \Tilde{H} \ket{\Tilde{v}'\Tilde{c}'} \bra{\Tilde{v}'\Tilde{c}'}\ket{v'c'} = \sum_{\Tilde{v}\Tilde{c}} \bra{vc}\ket{\Tilde{v}\Tilde{c}} (\epsilon_{\Tilde{v}} - \epsilon_{\Tilde{c}}) \bra{\Tilde{v}\Tilde{c}}\ket{v'c'} +  \Xi_{vc}^{v'c'} 
\end{equation}
Next, we use Eq. \eqref{eq:perturb_wf} to expand $\sum_{\Tilde{v}\tilde{c}} \bra{vc}\ket{\Tilde{v}\Tilde{c}} (\epsilon_{\Tilde{c}} - \epsilon_{\Tilde{v}})\bra{\Tilde{v}\Tilde{c}}\ket{v'c'}$ to order $\mathcal{O}(\Delta)$. We work within the Tamm-Dancoff approximation and keep only the resonant part of the BSE Hamiltonian; as a consequence, only valence-valence hole-phonon and conduction-conduction electron-phonon scattering are allowed, that is to say $\Delta_{vc} = \Delta_{cv} = 0$ where the operator $\Delta$ was defined in Eq. \eqref{eq:Delta_var}. Using Eq. \eqref{eq:perturb_wf} we get :
\begin{align}
\begin{split}
    \bra{vc}\ket{\tv\tc} &= \bra{v}\ket{\tv}\bra{c}\ket{\tc} = \left(\delta_{v\tv} + \sum_{v''\neq\tv}\Delta_{\tv v''}\delta_{vv''}\right) \left(\delta_{c\tc} + \sum_{c''\neq \tc}\Delta_{\tc c''}\delta_{cc''}\right) \\
    &= \left( \delta_{v\tv}\delta_{c\tc} + \delta_{v\tv} \sum_{c''\neq\tc} \Delta_{\tc c''}\delta_{cc''} + \delta_{c\tc}\sum_{v''\neq \tv} \Delta_{\tv v''} \delta_{vv''} \right) + \mathcal{O}(\Delta^2)
\end{split}
\end{align}
and similarly
\begin{equation}
    \bra{\tv\tc}\ket{v'c'} = \bra{v'}\ket{\tv}^*\bra{c'}\ket{\tc}^* = \left( \delta_{v'\tv}\delta_{c'\tc} + \delta_{v'\tv} \sum_{c''\neq\tc} \Delta^*_{\tc c''}\delta_{c'c''} + \delta_{c'\tc}\sum_{v''\neq \tv} \Delta^*_{\tv v''} \delta_{v'v''}  \right)  + \mathcal{O}(\Delta^2)
\end{equation}
With these expressions, there are five first-order terms in $\sum_{\Tilde{v}\tilde{c}} \bra{vc}\ket{\Tilde{v}\Tilde{c}} (\epsilon_{\Tilde{c}} - \epsilon_{\Tilde{v}})\bra{\Tilde{v}\Tilde{c}}\ket{v'c'}$ that we can simplify using the Kronecker delta :
\begin{align}
    \sum_{\tv\tc} \bra{vc}\ket{\tv\tc} (\epsilon_{\Tilde{c}} - \epsilon_{\Tilde{v}})\bra{\tv\tc}\ket{v'c'} \nonumber \\
    %
        \approx (\epsilon_c - \epsilon_v)\delta_{vv'}\delta_{cc'} &+ \delta_{cc'}\sum_{\tv}(\epsilon_c - \epsilon_{\tv}) \sum_{v''\neq \tv} (\Delta^*_{\tv v''}\delta_{vv''}\delta_{v'\tv} + \Delta_{\tv v''}\delta_{v'v''}\delta_{v\tv}) \nonumber \\
        &+  \delta_{vv'} \sum_{\tc} (\epsilon_{\tc} - \epsilon_v) \sum_{c'' \neq\tc} (\Delta_{\tc c''}\delta_{cc''}\delta_{c'\tc} + \Delta^*_{\tc c''}\delta_{c'c''}\delta_{c\tc}) \nonumber \\
        %
        = (\epsilon_c - \epsilon_v)\delta_{vv'}\delta_{cc'} &+ \delta_{cc'} \left[  \sum_{v''\neq v'} (\epsilon_c - \epsilon_{v'}) \Delta^*_{v'v''}\delta_{v v''} + \sum_{v'' \neq v} (\epsilon_c - \epsilon_v) \Delta_{vv''} \delta_{v'v''} \right] \nonumber \\
        &+  \delta_{vv'} \left[ \sum_{c''\neq c'} (\epsilon_{c'} - \epsilon_v)\Delta_{c' c''}\delta_{cc''} + \sum_{c''\neq c} (\epsilon_c - \epsilon_v) \Delta^*_{cc''}\delta_{c'c''} \right] \nonumber \\
    %
    = (\epsilon_c - \epsilon_v) \delta_{vv'}\delta_{cc'} &+ \delta_{cc'}(\epsilon_{v'} - \epsilon_v) \Delta_{vv'} + \delta_{vv'}(\epsilon_c - \epsilon_{c'}) \Delta^*_{cc'}
\end{align}
where we used $\Delta_{ij} = -\Delta_{ji}^*$ to obtain the last line. Finally, the perturbed Hamiltonian in the excitonic basis in Eq. \eqref{eq:exc_ham} becomes :
\begin{align}
    \tilde{H}_{\lambda\lambda'} &= \sum_{vc,v'c'} A_{\lambda'}^{vc*} A_{\lambda}^{v'c'} \times \left\{ \left[ (\epsilon_c - \epsilon_v) \delta_{vv'}\delta_{cc'} + \Xi_{vc}^{v'c'} \right] + \delta_{cc'} (\epsilon_{v'} - \epsilon_v)\Delta_{vv'} + \delta_{vv'}(\epsilon_c - \epsilon_{c'}) \Delta_{cc'}^*  \right\} \nonumber \\
    &= E_{\lambda'}\delta_{\lambda\lambda'} + \sum_{vc,v'c'} A_{\lambda'}^{vc*} A_{\lambda}^{v'c'} \cdot \left( \delta_{cc'}(\epsilon_{v'} - \epsilon_v) \Delta_{vv'}  + \delta_{vv'} (\epsilon_c - \epsilon_{c'}) \Delta^*_{cc'} \right) \label{eq:perturb_H_exc}
\end{align}
where we use the fact that the unperturbed Hamiltonian is diagonalized by the Tamm-Dancoff exciton eigenvectors :
\begin{equation}
    E_{\lambda'}\delta_{\lambda\lambda'} = \sum_{vc,v'c'} A_{\lambda'}^{vc*} A_{\lambda}^{v'c'} \times \left( (\epsilon_c - \epsilon_v)\delta_{vv'}\delta_{cc'} + \Xi_{vc}^{v'c'} \right).
\end{equation}
Therefore, the first term in the second line of Eq. \eqref{eq:perturb_H_exc} is the unperturbed Hamiltonian, while the second term is the exciton-phonon interaction,
\begin{equation}
    \tilde{H}_{\text{exc-ph}} = \sum_{vc,v'c'} A_{\lambda'}^{vc*} A_{\lambda}^{v'c'} \cdot \left( \delta_{cc'}(\epsilon_{v'} - \epsilon_v) \Delta_{vv'}  + \delta_{vv'} (\epsilon_c - \epsilon_{c'}) \Delta^*_{cc'} \right). \label{eq:H_exc-ph}
\end{equation}
To obtain the final result, we reintroduce the momemtum dependence and the Bloch states :
\begin{equation}
    \ket{\phi_i} \to \ket{\phi_{n\kk}}
\end{equation}
and the transition basis set for an exciton with center of mass momentum $\QQ$ is $\ket{vc} = \ket{v\kk_v,c\kk_c} = \ket{v\kk_v,c\kk_v + \QQ}$. We write the change in potential due to atomic displacements using the phonon normal coordinates :
\begin{equation}
    \Delta V = \sum_{\mu \qq} \sqrt{\frac{1}{2\Omega_{\mu\qq}}} \partial_{\mu\qq} V^{KS}(\hat{b}_{\mu\qq} + \hat{b}^\dagger_{\mu-\qq})  
\end{equation}
where the operator $\partial_{\mu\qq}$ should be understood as the derivative with respect to a displacement along a phonon mode $\mu$ at momentum $\qq$. Then the $\Delta_{ij}$ describing the transition from $i$-th to $j$-th state becomes :
\begin{equation}
    \Delta_{n\kk n'\kk'} = \frac{\bra{n'\kk'} \Delta V \ket{n\kk}}{\epsilon_{n\kk} - \epsilon_{n'\kk'}} = \sum_{\mu\qq} \frac{g_{nn'\mu}(\kk,\qq) \delta(\kk'-\kk-\qq)}{\epsilon_{n\kk} - \epsilon_{n'\kk'}} (\hat{b}_{\mu\qq} + \hat{b}^\dagger_{\mu-\qq})
\end{equation}
where $g_{nn'\mu}(\kk,\qq)$ are the electron-phonon matrix elements defined in Eq. \eqref{eq:gkkp}, namely the probability amplitude for an electron in band $n$ with crystal momentum $\kk$ to transition to a final state in band $n'$ and momentum $\kk' = \kk+\qq$ by absorbing or emitting a phonon with mode index $\mu$ and wave vector $\qq$. The slight difference with Eq. \eqref{eq:gkkp} is that we change the arguments to have a more compact form : the first argument is the momentum of the initial state and the second argument is the phonon momentum.

By introducing exciton creation and annihilation operators, $\hat{a}^\dagger_{\lambda(\QQ)}$ and $\hat{a}_{\lambda(\QQ)}$, we rewrite the exciton-phonon interaction from Eq. \eqref{eq:H_exc-ph} as :
\begin{equation}
    \tilde{H}_{\text{exc-ph}} = \sum_{\lambda\lambda'\mu, \QQ\qq} \mathcal{G}_{\lambda\lambda'}^{\mu}\QQ,\qq \hat{a}^\dagger_{\lambda\QQ+\qq} \hat{a}_{\lambda\QQ} (\hat{b}_{\mu\qq} + \hat{b}^\dagger_{\mu-\qq}).
\end{equation}
where we defined the exciton-phonon matrix elements as :
\begin{multline}
    \mathcal{G}_{\lambda\lambda'}^{\mu}(\QQ,\qq) = \sum_{\substack{vcv'c'\\ \kk_v \kk_c \kk'_{v'} \kk'_{c'}}} A_{\lambda\QQ+\qq}(v\kk_v,c\kk_c) A_{\lambda'\QQ}^*(v'\kk'_{v'},c'\kk'_{c'}) \\ 
    \times \left[ \delta_{vv'} g_{c'c\mu}(\kk'_{c'},\qq) \delta(\kk_c - \kk'_{c'} - \qq) \right. \left. - \delta_{cc'}g_{vv'\mu}(\kk_v,\qq) \delta(\kk'_{v'} - \kk_v -\qq) \right]. \label{eq:Gkkp}
\end{multline}
Let us make momentum conservation explicit to obtain the final expression. The exciton-phonon coupling constant $\mathcal{G}_{\lambda\lambda'\mu}(\QQ,\qq)$ is the probability amplitude for scattering from an exciton with band index $\lambda$ with center-of-mass momentum $\QQ + \qq$ to an exciton with band index $\lambda'$ and center-of-mass momentum $\QQ$. This convention will be clarified later. Since $A_{\lambda\QQ}(v\kk_{v},c\kk_{c}) \neq 0$ only for $\kk_c - \kk_v = \QQ$, in Eq. \eqref{eq:Gkkp} we can impose three constraints : $\kk_c - \kk_v = \QQ$, $\kk'_c - \kk'_v =\QQ + \qq$ and $\kk'_c - \kk_c = \qq$ (or $\kk'_v - \kk_v = \qq$). As a consequence, we drop three $\kk$-point \acrshort{BZ} summations and the final result for the exciton-phonon matrix element for a given exciton momentum $\QQ$ and phonon momentum $\qq$ is :
\begin{multline}
    \mathcal{G}_{\lambda\lambda'}^{\mu}(\QQ,\qq) \\
    = \sum_{\kk} \left[ \sum_{vcc'} A_{\lambda\QQ+\qq}(v\kk,c\kk+\QQ+\qq) A_{\lambda'\QQ}^{*}(v\kk,c'\kk+\QQ)g_{c'c\mu}(\kk+\QQ +\qq,-\qq) \right.\\
     \left. - \sum_{cvv'} A_{\lambda\QQ+\qq}(v\kk-\qq,c\kk+\QQ) A_{\lambda'\QQ}^*(v'\kk,c\kk+\QQ) g_{vv'\mu}(\kk-\qq,\qq) \right].
\end{multline}
\textcolor{red}{we took this convention for the scattering because final state is a direct exciton}


\section{PL benchmark on hBN and results for mBN}
better than Chen but intensity reversed for LO/TO. Mention that in Matteo's paper they have the correct intensities by using Lbar at Gamma and Lfull otherwise I think ? \\
experimental spectra

\section{effect of the substrate}
electronic gap, distortion of excitonic dispersion to simulate a change in the screening.

\section{Preliminary results on bBN}
show the distorted spectrum, which also contains the ZO peaks



