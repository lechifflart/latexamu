\chapter*{Conclusion}
\addcontentsline{toc}{chapter}{Conclusion}

The scope of this thesis was to reproduce and predict the luminescence spectra of distinct hBN nanostructures. Additionally it provides an insightful interpretation of the different features appearing in experimental measurements thanks to first-principles calculations of the exciton-phonon coupling. These calculations are built upon the state-of-the-art \textit{ab initio} theories presented in the second Chapter.

\section*{Main results}
In the third Chapter, we applied an existing approach where the response function gets a static correction due to the phonon-assisted transitions. This correction is calculated through finite-difference derivatives, wherein we build supercells and displace the contained atoms along all the phonon modes. We applied this method to compute the luminescence spectra of bulk hBN under uniaxial strain. We found that multiple nonequivalent excitonic valleys in the Brillouin Zone contribute to luminescence due to rotational symmetry breaking.
Therefore the phonon-assisted peaks split for low values of strain. This splitting also broadens the peaks and increases the intensity of the LA/TA doublet in relation to the LO/TO.
This partially agrees with recent experimental measurements of cathodoluminescence of strained \acrshort{hBN}.\\

In the fourth Chapter, we went beyond the finite-difference method and presented the derivation of an \textit{ab initio} method to obtain exciton-phonon coupling.This method enables the treatment of the coupling of all excitons with all phonon modes in the entire Brillouin Zone and represents an improvement in both numerical and theoretical aspects. It yields the renormalization of the direct peak by the appearance of phonon-assisted satellites in the luminescence spectrum, originating from the dynamical correction to the response function. 
We implemented this approach in the \texttt{yambo} code and applied it to the calculation of phonon-assisted luminescence of
a monolayer of hBN. We obtained the first results of calculated luminescence from first-principles, encompassing both direct and phonon-assisted emission signatures. We were able to compare their relative intensities and ruled out the possibility of distinguishing phonon satellites due to the bright intensity of the direct peak. This possibility was one of the interpretations proposed in very recently published experimental measurements displaying features of debated origin. 
We also investigated the role of the screening of a graphitic substrate on the intensity of the indirect exciton peak and found that it is likely not sufficient to reveal the phonon-assisted processes in the experiments. \\

We finally showed the preliminary results of an ongoing study on the Bernal phase of BN. Theoretical and experimental evidence suggest that both direct and phonon-assisted emission processes could be visible with comparable intensities. This makes it an ideal testing ground for our approach. Initial results suggest a direct luminescence but it may vary by carefully tuning the numerical parameters.

\section*{Future work and perspectives}
Regarding the results of strained \acrshort{hBN}, the agreement with experimental data could be improved by sampling the excitonic dispersion with a finer grid in reciprocal space. A tight-binding model adjusted on coarse \textit{ab initio} data could help refine this sampling. Moreover, the strain could be included as a parameter of the model. Experimentally, the strain applied to the sample could be estimated with reflectance and optical absorption experiments. These could be compared with our results for the variation of exciton energy and the splitting of the absorption main peak. This way a joint experimental and theoretical study on the effect of strain on optical properties of \acrshort{hBN} could be performed. \\

Regarding the \textit{ab initio} calculation of the exciton-phonon coupling, a workaround to the wrong intensities and to the phase problem was found by Zanfrognini \textit{et al} in Ref. \cite{zanfrognini2023distinguishing}. However their approach necessitates the use of a third simulation code and does not exploits the crystal symmetries to reduce the size of the Brillouin Zone. This significantly increases the computational load and the disk space necessary for the workflow. A promising solution to the phase problem is currently being developed, which involves a different way of calculating the electron-phonon interaction.\\

For the study of the Bernal phase of BN, we are investigating ways to reduce the disk space required, which can be prohibitive for more complex materials, without compromising the accuracy of the workflow. The study of \acrshort{bBN} and the dependence of its luminescence spectrum with respect to the position of the excitonic valleys could pave the way for engineering its optical properties and potentially build devices that take advantage of the direct and indirect nature of this material.\\ 

Although in this thesis the exciton-phonon coupling has been calculated with two different methods, the luminescence intensity has been obtained using the van Roosbroeck--Shockley relation in both cases. On the one hand, assuming we are in a steady state situation, it allows to compute the luminescence from the knowledge of the macroscopic dielectric function that we obtain \textit{ab initio}. It is much simpler both numerically and theoretically than considering the out-of-equilibrium interplay between electrons, holes, phonons and photons. On the other hand, it requires an external parameter, the excitonic temperature, hence the workflow is not fully \textit{ab initio}. It should be possible to avoid this and derive an exciton conversion rate from the exciton-phonon coupling we have computed. We could then include the exciton relaxation from higher energy to the lower valley and their thermalization and replace the \textit{ad hoc} excitonic temperature by an \textit{ab initio} one.\\

In the current state, the knowledge of the \textit{ab initio} exciton-phonon matrix elements can be used to study exciton dynamics.\cite{cohen2023phonon} The exciton-phonon self-energy also allows to compute the exciton lifetimes which are helpful for the ultrafast spectroscopy, a field of relevance in Condensed Matter Physics.\\

Moreover, this \textit{ab initio} method could be extended to the scattering with multiple phonons, also including renormalization of the exciton energies by dynamical interaction with phonons, thanks to the cumulant expansion as shown theoretically by Cudazzo in Ref. \cite{cudazzo2020first}. Our implementation in the \yambo~code can be readily extended to this more general case. It would then allow one to simulate the temperature dependence of the exciton energies and of the optical spectra on the same footing.\\


Finally, a rigorous treatment of the exciton-lattice interaction is the first step towards a description of the excited state geometry of periodic solids. It could used to the calculation of forces in a GW-BSE framework for instance, with applications to exciton self-trapping or charge separation, essential phenomena for optoelectronic devices and photovoltaics.

