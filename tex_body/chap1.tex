\linenumbers
\chapter{ State of the art theory}
\chaptertoc{}

\section{Introduction on the many-body problem}
%generalities on theories that are the building blocks of our workflows
%what is ab initio
The system of interacting electrons and nuclei in a material is described by the following Hamiltonian :
\begin{align}
	\hat{H} = &- \sum_I \frac{\hbar^2}{2 M_I} \nabla^2_I + \frac{1}{2} \sum_{I\neq J} \frac{Z_I Z_J e^2}{\left| \RR_I - \RR_J \right|} \nonumber \\
	&- \frac{\hbar^2}{2m_e} \sum_i \nabla^2_i + \frac{1}{2} \sum_{i\neq j} \frac{e^2}{\left| \rr_i - \rr_j \right|} - \sum_{i,I} \frac{Z_I e^2}{\left|\rr_i - \RR_I \right|}
\end{align}
where capital indices refer to nuclei and lowercase indices refer to electrons. $Z_I$, $M_I$ and $\RR_I$ are the atomic number, the mass and the position in real space of nucleus $I$. In this thesis, unless explicitly specified, we will use the atomic units, that is $\hbar = m_e = e =\pi/\varepsilon_0 =1$.
The terms on the first line are the kinetic energy of the ions and the ion-ion interaction, respectively. On the second line we have the kinetic energy of electrons, that we will later denote $T_{ee}$, the electron-electron interaction $V_{ee}$ and finally the electron-nucleus interaction, which we will refer to as the external potential felt by electrons in equilibrium, $V_{ext}$. To greatly simplify this Hamiltonian, we use the Born-Oppenheimer approximation. It consists in considering the nuclei fixed, because their masses are much greater than those of the electrons. The Hamiltonian then reads :
\be
	\hat{H} = \hat{T_e} + \hat{V}_{ee} + \hat{V}_{ext}
\ee
The first and second term are universal for all systems.The peculiarities of any systems are included in the last term in the above equation.

\section{Density Functional Theory}
This theory is vastly used in solid-state physics and quantum chemistry. In this thesis, it will be the starting tool to compute structural and electronic properties of our systems. The idea behind \gls{DFT} is to replace the real system of interacting electrons by an auxiliary system of non-interacting particles evolving in an effective potential.

\gls{DFT} is a theory of the interacting electron system and is based on the work of Hohenberg and Kohn who stated and proved two fundamental theorems. \cite{hohenberg1964} The first one ensures there is a one-to-one correspondance between the electronic density and the external potential acting on the system. The second theorem states that the total energy of the system is a functional of the electronic density.
The total energy of a system of interacting electrons is written as :
\be
	E = \expval{\hat{H}}{\Psi} = \expval{\hat{T} + \hat{V}_{ee} }{\Psi} + \int d\rr v_{ext}(\rr)n(\rr)
\ee
By virtue of the Hohenberg and Kohn theorems, the total energy is a functional of the density and can be written as :
\be
 	E_{HK}[n] = F_{HK}[n] + \int d\rr v_{ext}(\rr)n(\rr)
	\label{eq:E_HK}
\ee
where $F_{HK}[n] = \la \hat{T} \ra + \la \hat{V}_{ee} \ra$ is a universal functional of the density, \emph{ie} the dependence on $n$ of the functional is the same for all systems. The ground-state energy $E = E_0$ is the minimum of the energy functional at the ground-state density $n=n_0$. To be able to compute these quantities, Kohn and Sham reformulated the problem into an auxiliary system of non-interacting particles, that has the same density as the real system.\cite{kohn1965} This reformulation is particularly helpful because it is possible to compute analytically the kinetic energy term. The only subtlety is that one needs to use the density \emph{matrix} rather than the electronic density, that is defined as the expectation value of the density operator $\hat{\delta}(\rr) = \sum_i^{N_{e}} \delta(\rr - \rr_i)$ :
\begin{align}
\begin{split}
	n(\rr) &= \int d^3r_1 ... d^3r_{N_{e}} \Psi^*(\rr_1,...,\rr_{N_{e}})
	\hat{\delta}(\rr) \Psi(\rr_1,...,\rr_{N_{e}}) \\
	&= N_{e}\int d^3r_2 ... d^3r_{N_{e}} \left| \Psi(\rr,\rr_2...,\rr_{N_{e}})\right|^2
\end{split}
\end{align}
This is related to the probability density of finding a particle at position $\rr$.
The expression for the total energy functional in \ref{eq:E_HK} can be rewritten as :
\be
	E_{KS}[n] = T_{ip}[n] + \int d\rr v_{ext}(\rr)n(\rr) + E_H[n] + E_{XC}[n]
\ee
where $T_{ip}$ is the kinetic energy of the independent particles with density $n$, $E_H$ is the Hartree energy, which is the classical electrostatic interaction :
\be
	E_H[n] = \int d\rr d\rr' \frac{n(\rr')n(\rr)}{\left|\rr - \rr' \right|}
	\label{eq:E_Hartree}
\ee
and $E_{XC}$ is the exchange-correlation energy functional defined as :
\be
	E_{XC}[n] = \langle \hat{T} \rangle - T_{ip} + \langle \hat{V}_{ee} \rangle - E_H.
\ee
It is the difference between the exact kinetic energy and $T_{ip}$ plus the difference between the exact electron-electron interaction and the Hartree energy functional. Hence it contains the quantum effects of exchange and correlation of fermions.

Since the auxiliary system is an ensemble of independent particles, one can write the so-called Kohn-Sham equations for each individual particle $i$ :
\be
 	\( -\frac{\nabla^2}{2} + v_{eff}(\rr)\) \psi_i(\rr) = \varepsilon_i \psi_i(\rr).
\ee
They are analogous to Schrödinger equation for a particle evolving in a local effective potential $v_{eff}$, that we have yet to determine. Their solutions are the auxiliary system's eigenvalues $\varepsilon_i$ and the eigenvectors $\psi_i$ that are defined as :
\be
 	n(\rr) = \sum_i f_i \left| \psi_i\right|^2
\ee
where $f_i$ is the occupation number of state $i$. Using the fact that the total kinetic energy is independent of the density at fixed number of particles, we get :
\begin{align}
\begin{split}
	v_{eff}(\rr) &= v_{ext}(\rr) + \frac{\delta E_H[n]}{\delta n(\rr)} + \frac{\delta E_{xc}[n]}{\delta n(\rr)} \\
	&\equiv v_{ext}(\rr) + v_H([n],\rr) + v_{xc}([n],\rr).
\end{split}
\end{align}
At this point, we see that we have to solve the many-electron problem self-consistently. Indeed, the density is obtained by solving the Kohn-Sham equations which contain the effective potential. In turn, this potential depends on the density. In practice, one starts from a guess density and iterate over the self-consistent cycle until the quantities of interest vary less than an arbitrary threshold.

Up to now, \gls{DFT} is in principle an exact theory, as long as one can define an auxiliary system with the same density as the real system. However, there exists no analytical form of the exchange-correlation potential. Hence we will then have to resort to approximations to compute the density in practice.

The \textbf{\gls{LDA}} is the first one that we present, and that we used for most of the results in this thesis. It was proposed by Kohn and Sham \cite{kohn1965}. It consists in replacing the exchange-correlation energy density by the one of the homogeneous electron gas, which is local in the density :
\be
	E_{XC}^{LDA} [n] = \int d\rr \ \epsilon^{HEG}_{xc}(n(\rr))
\ee
The exchange energy density of the homogeneous electron gas is known : $\epsilon_X^{HEG}(n) = -\tfrac{3}{4}(\tfrac{3}{\pi}n)^{(1/3)}$, and the correlation energy density is obtained from interpolation of an accurate quantum Monte Carlo simulation \cite{ceperley1980ground} for various values of densities. The LDA is relatively simple and computationally inexpensive. It gives a satisfactory description of system with slowly-varying density, but also surprisingly good results for a larger range of materials. For instance, for layered materials such as hBN, the interlayer binding energies are rather accurate, due to the tendency of overbinding of the LDA which cancels the error induced by the lack of van der Waals interactions.

In the \textbf{\gls{GGA}} the exchange-correlation energy density contains an additional dependence in the gradient of the density :
\be
 	E_{XC}^{GGA} [n] = \int d\rr \ \epsilon^{GGA}_{xc}(n(\rr),\nabla n(\rr)).
\ee
There are many different ways of expressing the exchange-correlation energy density in this case. Some of the proposed approximated functionals include the exact exchange term or a fraction of it, and it is separated from the correlation term.
There are many more approximations going beyond LDA and GGA functionals that we will not detail here, as we did not use them for the DFT calculations in this thesis.\newline

In our work, the description of each and every electron of the crystal is not necessary. Indeed, for the range of energies we are interested in, the electrons that can be optically excited are the ones that occupied the highest valence shells. The core electrons of the first $s$ and $p$ shells are bound too strongly to the nuclei to be excited by a few electron-Volts incoming light. Hence, to simplify the system, we use pseudopotentials to avoid describing the core electrons.
Pseudopotentials solve the problem of the divergence of the Coulomb potential as $\rr  \to 0$, which leads to rapid oscillations in the wave functions of the valence orbitals.
Beyond a given cutoff radius $r_c$, pseudopotentials are constrained to be exactly equal to the true
potential. Below this radius, the pseudopotential does not diverge and assumes a finite value at $r=0$. This generates a pseudo-wave function which is smooth and does not oscillate below the cutoff radius. With this method, the Kohn-Sham eigenvalues remain unchanged, and the computationally demanding task of representing the oscillating wave function is eliminated.\newline

For real systems, the wave function of the crystal is an immensely complicated object whose analytical expression is out of reach. For computational purposes, one needs to represent it in a complete basis of the Hilbert space. Depending on the characteristics of the system, the choice of the basis functions is different.

In this thesis we study infinite, periodic crystals. A suitable basis for this case is the plane waves. It is based on the Bloch theorem, which states that the square modulus of the wave function of an electron has the same periodicity than the crystal potential :
\be
 	\phi_{j,\kk} (\rr) = u_j (\rr) e^{i\kk\cdot\rr}
\ee
The $u_j$ functions have the same periodicity as the lattice. They are the one we decompose into plane waves as such :
\be
 	u_j (\rr) = \sum_{\GG} c_{j,\GG} e^{i\GG \cdot \rr}
\ee
where $c_{j,\GG}$ are the coefficients of the plane waves basis, and the $\GG$ are the reciprocal lattice vectors. The plane waves expansion of the band $j$ follows :
\be
	\phi_{j,\kk} (\rr) = \sum_{\GG} c_{j,\kk+\GG} e^{i (\kk+\GG)\cdot \rr}
\ee
In principle all complete bases of the Hilbert space yield equal representation of the wave function. As it is impossible to numerically sample a Hilbert space with infinite dimension, one has to truncate the representation. Thus, one has to make sure enough basis functions are included in the expansion to have an accurate wave function.
In this case, we choose an energy cutoff $E_{cut}$ such that :
\be
	\frac{1}{2} \left| \kk+\GG \right|^2 \leq E_{cut}
\ee
One has to verify that the cutoff is high enough so that the results are accurate, but setting it too high would mean including more plane waves which would slow down the calculations. \newline

Density Functional Theory is often used as a reference for bandgaps and electronic dispersion calculations. One has to be careful when interpreting Kohn-Sham eigenvalues, because they do not bear any physical meaning. First, there is no guaranty that one can find an auxiliary system of non-interacting particles for any real system. Then, the excited states and bandgaps are not the physical ones. This is especially relevant if one wants to simulate optics experiments of semiconductors or insulators. Optical excitations are neutral excitations, in the sense that the excited electron does not leave the system and therefore can interact with the hole it left behind in the valence band. This interaction between a hole in a valence band and an excited electron in the conduction bands is not accurately accounted for in \gls{DFT}, as it is designed to compute total energy and electronic density of the \textit{groundstate}. For this reason, we will resort to a more refined theoretical framework to treat the electronic correlations and study the excited states.




%(Structure optimization ? Hellmann-Feynman theorem to compute the forces acting on atoms, compute the total energy, minimize both the forces and the total energy)
