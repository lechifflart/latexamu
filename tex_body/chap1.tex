\linenumbers
\chapter{ State of the art theory}
\chaptertoc{}

\section{Introduction on the many-body problem}
%generalities on theories that are the building blocks of our workflows
%what is ab initio
The system of interacting electrons and nuclei in a material is described by the following Hamiltonian :
\begin{align}
	\hat{H} = &- \sum_I \frac{\hbar^2}{2 M_I} \nabla^2_I + \frac{1}{2} \sum_{I\neq J} \frac{Z_I Z_J e^2}{\left| \RR_I - \RR_J \right|} \nonumber \\
	&- \frac{\hbar^2}{2m_e} \sum_i \nabla^2_i + \frac{1}{2} \sum_{i\neq j} \frac{e^2}{\left| \rr_i - \rr_j \right|} - \sum_{i,I} \frac{Z_I e^2}{\left|\rr_i - \RR_I \right|}
\end{align}
where capital indices refer to nuclei and lowercase indices refer to electrons. $Z_I$, $M_I$ and $\RR_I$ are the atomic number, the mass and the position in real space of nucleus $I$. In this thesis, unless explicitly specified, we will use the atomic units, that is $\hbar = m_e = e =\pi/\varepsilon_0 =1$.
The terms on the first line are the kinetic energy of the ions and the ion-ion interaction, respectively. On the second line we have the kinetic energy of electrons, that we will later denote $T_{ee}$, the electron-electron interaction $V_{ee}$ and finally the electron-nucleus interaction, which we will refer to as the external potential felt by electrons in equilibrium, $V_{ext}$. To greatly simplify this Hamiltonian, we use the Born-Oppenheimer approximation. It consists in considering the nuclei fixed, because their masses are much greater than those of the electrons. The Hamiltonian then reads :
\be
	\hat{H} = \hat{T_e} + \hat{V}_{ee} + \hat{V}_{ext}
\ee
The first and second term are universal for all systems.The peculiarities of any systems are included in the last term in the above equation.

\section{Density Functional Theory}
This theory is vastly used in solid-state physics and quantum chemistry. In this thesis, it will be the starting tool to compute structural and electronic properties of our systems. The idea behind \gls{DFT} is to replace the real system of interacting electrons by an auxiliary system of non-interacting particles evolving in an effective potential.

\gls{DFT} is a theory of the interacting electron system and is based on the work of Hohenberg and Kohn who stated and proved two fundamental theorems. \cite{hohenberg1964} The first one ensures there is a one-to-one correspondance between the electronic density and the external potential acting on the system. The second theorem states that the total energy of the system is a functional of the electronic density.
The total energy of a system of interacting electrons is written as :
\be
	E = \expval{\hat{H}}{\Psi} = \expval{\hat{T} + \hat{V}_{ee} }{\Psi} + \int d\rr v_{ext}(\rr)n(\rr)
\ee
By virtue of the Hohenberg and Kohn theorems, the total energy is a functional of the density and can be written as :
\be
 	E_{HK}[n] = F_{HK}[n] + \int d\rr v_{ext}(\rr)n(\rr)
	\label{eq:E_HK}
\ee
where $F_{HK}[n] = \la \hat{T} \ra + \la \hat{V}_{ee} \ra$ is a universal functional of the density, \emph{ie} the dependence on $n$ of the functional is the same for all systems. The ground-state energy $E = E_0$ is the minimum of the energy functional at the ground-state density $n=n_0$. To be able to compute these quantities, Kohn and Sham reformulated the problem into an auxiliary system of non-interacting particles, that has the same density as the real system.\cite{kohn1965} This reformulation is particularly helpful because it is possible to compute analytically the kinetic energy term. The only subtlety is that one needs to use the density \emph{matrix} rather than the electronic density, that is defined as the expectation value of the density operator $\hat{\delta}(\rr) = \sum_i^{N_{e}} \delta(\rr - \rr_i)$ :
\begin{align}
\begin{split}
	n(\rr) &= \int d^3r_1 ... d^3r_{N_{e}} \Psi^*(\rr_1,...,\rr_{N_{e}})
	\hat{\delta}(\rr) \Psi(\rr_1,...,\rr_{N_{e}}) \\
	&= N_{e}\int d^3r_2 ... d^3r_{N_{e}} \left| \Psi(\rr,\rr_2...,\rr_{N_{e}})\right|^2
\end{split}
\end{align}
This is related to the probability density of finding a particle at position $\rr$.
The expression for the total energy functional in \ref{eq:E_HK} can be rewritten as :
\be
	E_{KS}[n] = T_{ip}[n] + \int d\rr v_{ext}(\rr)n(\rr) + E_H[n] + E_{XC}[n]
\ee
where $T_{ip}$ is the kinetic energy of the independent particles with density $n$, $E_H$ is the Hartree energy, which is the classical electrostatic interaction :
\be
	E_H[n] = \int d\rr d\rr' \frac{n(\rr')n(\rr)}{\left|\rr - \rr' \right|}
	\label{eq:E_Hartree}
\ee
and $E_{XC}$ is the exchange-correlation energy functional defined as :
\be
	E_{XC}[n] = \langle \hat{T} \rangle - T_{ip} + \langle \hat{V}_{ee} \rangle - E_H.
\ee
It is the difference between the exact kinetic energy and $T_{ip}$ plus the difference between the exact electron-electron interaction and the Hartree energy functional. Hence it contains the quantum effects of exchange and correlation of fermions.

Since the auxiliary system is an ensemble of independent particles, one can write the so-called Kohn-Sham equations for each individual particle $i$ :
\be
 	\( -\frac{\nabla^2}{2} + v_{eff}(\rr)\) \psi_i(\rr) = \varepsilon_i \psi_i(\rr).
\ee
They are analogous to Schrödinger equation for a particle evolving in a local effective potential $v_{eff}$, that we have yet to determine. Their solutions are the auxiliary system's eigenvalues $\varepsilon_i$ and the eigenvectors $\psi_i$ that are defined as :
\be
 	n(\rr) = \sum_i f_i \left| \psi_i\right|^2
\ee
where $f_i$ is the occupation number of state $i$. Using the fact that the total kinetic energy is independent of the density at fixed number of particles, we get :
\begin{align}
\begin{split}
	v_{eff}(\rr) &= v_{ext}(\rr) + \frac{\delta E_H[n]}{\delta n(\rr)} + \frac{\delta E_{xc}[n]}{\delta n(\rr)} \\
	&\equiv v_{ext}(\rr) + v_H([n],\rr) + v_{xc}([n],\rr).
\end{split}
\end{align}
At this point, we see that we have to solve the many-electron problem self-consistently. Indeed, the density is obtained by solving the Kohn-Sham equations which contain the effective potential. In turn, this potential depends on the density. In practice, one starts from a guess density and iterate over the self-consistent cycle until the quantities of interest vary less than an arbitrary threshold.

Up to now, \gls{DFT} is in principle an exact theory, as long as one can define an auxiliary system with the same density as the real system. However, there exists no analytical form of the exchange-correlation potential. Hence we will then have to resort to approximations to compute the density in practice.

The \textbf{\gls{LDA}} is the first one that we present, and that we used for most of the results in this thesis. It was proposed by Kohn and Sham \cite{kohn1965}. It consists in replacing the exchange-correlation energy density by the one of the homogeneous electron gas, which is local in the density :
\be
	E_{XC}^{LDA} [n] = \int d\rr \ \epsilon^{HEG}_{xc}(n(\rr))
\ee
The exchange energy density of the homogeneous electron gas is known : $\epsilon_X^{HEG}(n) = -\tfrac{3}{4}(\tfrac{3}{\pi}n)^{(1/3)}$, and the correlation energy density is obtained from interpolation of an accurate quantum Monte Carlo simulation \cite{ceperley1980ground} for various values of densities. The LDA is relatively simple and computationally inexpensive. It gives a satisfactory description of system with slowly-varying density, but also surprisingly good results for a larger range of materials. For instance, for layered materials such as hBN, the interlayer binding energies are rather accurate, due to the tendency of overbinding of the LDA which cancels the error induced by the lack of van der Waals interactions.

In the \textbf{\gls{GGA}} the exchange-correlation energy density contains an additional dependence in the gradient of the density :
\be
 	E_{XC}^{GGA} [n] = \int d\rr \ \epsilon^{GGA}_{xc}(n(\rr),\nabla n(\rr)).
\ee
There are many different ways of expressing the exchange-correlation energy density in this case. Some of the proposed approximated functionals include the exact exchange term or a fraction of it, and it is separated from the correlation term.
There are many more approximations going beyond LDA and GGA functionals that we will not detail here, as we did not use them for the DFT calculations in this thesis.\newline

In our work, the description of each and every electron of the crystal is not necessary. Indeed, for the range of energies we are interested in, the electrons that can be optically excited are the ones that occupied the highest valence shells. The core electrons of the first $s$ and $p$ shells are bound too strongly to the nuclei to be excited by a few electron-Volts incoming light. Hence, to simplify the system, we use pseudopotentials to avoid describing the core electrons.
Pseudopotentials solve the problem of the divergence of the Coulomb potential as $\rr  \to 0$, which leads to rapid oscillations in the wave functions of the valence orbitals.
Beyond a given cutoff radius $r_c$, pseudopotentials are constrained to be exactly equal to the true
potential. Below this radius, the pseudopotential does not diverge and assumes a finite value at $r=0$. This generates a pseudo-wave function which is smooth and does not oscillate below the cutoff radius. With this method, the Kohn-Sham eigenvalues remain unchanged, and the computationally demanding task of representing the oscillating wave function is eliminated.\newline

For real systems, the wave function of the crystal is an immensely complicated object whose analytical expression is out of reach. For computational purposes, one needs to represent it in a complete basis of the Hilbert space. Depending on the characteristics of the system, the choice of the basis functions is different.

In this thesis we study infinite, periodic crystals. A suitable basis for this case is the plane waves. It is based on the Bloch theorem, which states that the square modulus of the wave function of an electron has the same periodicity than the crystal potential :
\be
 	\phi_{j,\kk} (\rr) = u_j (\rr) e^{i\kk\cdot\rr}
\ee
The $u_j$ functions have the same periodicity as the lattice. They are the one we decompose into plane waves as such :
\be
 	u_j (\rr) = \sum_{\GG} c_{j,\GG} e^{i\GG \cdot \rr}
\ee
where $c_{j,\GG}$ are the coefficients of the plane waves basis, and the $\GG$ are the reciprocal lattice vectors. The plane waves expansion of the band $j$ follows :
\be
	\phi_{j,\kk} (\rr) = \sum_{\GG} c_{j,\kk+\GG} e^{i (\kk+\GG)\cdot \rr}
\ee
In principle all complete bases of the Hilbert space yield equal representation of the wave function. As it is impossible to numerically sample a Hilbert space with infinite dimension, one has to truncate the representation. Thus, one has to make sure enough basis functions are included in the expansion to have an accurate wave function.
In this case, we choose an energy cutoff $E_{cut}$ such that :
\be
	\frac{1}{2} \left| \kk+\GG \right|^2 \leq E_{cut}
\ee
One has to verify that the cutoff is high enough so that the results are accurate, but setting it too high would mean including more plane waves which would slow down the calculations. \newline

Density Functional Theory is often used as a reference for bandgaps and electronic dispersion calculations. One has to be careful when interpreting Kohn-Sham eigenvalues, because they do not bear any physical meaning. First, there is no guaranty that one can find an auxiliary system of non-interacting particles for any real system. Then, the excited states and bandgaps are not the physical ones. This is especially relevant if one wants to simulate optics experiments of semiconductors or insulators. Optical excitations are neutral excitations, in the sense that the excited electron does not leave the system and therefore can interact with the hole it left behind in the valence band. This interaction between a hole in a valence band and an excited electron in the conduction bands is not accurately accounted for in \gls{DFT}, as it is designed to compute total energy and electronic density of the \textit{groundstate}. For this reason, we will resort to a more refined theoretical framework to treat the electronic correlations and study the excited states.

%(Structure optimization ? Hellmann-Feynman theorem to compute the forces acting on atoms, compute the total energy, minimize both the forces and the total energy)

\section{Many-Body Perturbation Theory}
zero temperature
concept of quasiparticle
self-energy is difference between the independent particle energy and the quasiparticle energy (for the real part) ; the imaginary part is the inverse QP lifetime
define Green's function as electronic propagator
explain what we do in the next subsection; define G, derive its equation of motion, get $G_2$, nested dependence on higher-order GFs hence : resum with a self-energy. To compute self-energy, we need a cycle of equations with vertex, polarization etc; approximate self-energy $\Sigma = GW$ to simplify the Hedin's equations 
define screening
derive GW

\subsection{Hedin's equations}
% First, define the Green's function
\textcolor{red}{add more explanations, see Fulvio's thesis; especially correspondance between Schrödinger equation and GF}
We consider a system of N interacting electrons. We will consider the effect of nuclei motion in a later section.
We start from the Hamiltonian for interacting electrons in second quantization:
\begin{align}
\begin{split}
 	\hat{H} &= \int dx_1 \ \hpsidag(x_1) h(\rr_1) \hpsi(x_1) + \frac{1}{2} \iint dx_1 \  dx_2 \ \hpsidag(x_1) \hpsidag(x_2) v(\rr_1,\rr_2) \hpsi(x_2)\hpsi(x_1) \\
		&= \hat{H}_0 + \hat{H}_{int}.
\end{split}
\end{align}
where $\hat{\psi}(x)$ is a field operator in Schrödinger representation, $h$ is the single-particle Hamiltonian for non-interacting particles evolving in an external potential $V_{ext}$ and $v(\rr_1,\rr_2) = e^2 (\left| \rr_1 - \rr_2 \right|)^{-1}$ is the Coulomb interaction. In the above equation, x is the set of space and spin variables $x = (\rr, \sigma)$.
%

For the following derivation, we introduce an external potential $\Phi(x,x';t)$ which is local in time but nonlocal in space. We write it in the form of an interaction Hamiltonian :
\begin{equation}
	\hat{H}'(t) = \int dx dx' \hpsidag(x) \Phi(x,x';t) \hpsi(x') 
\end{equation}
In our case, this external perturbation will be set to 0 at the end of the calculation. It is a formal tool to derive the useful equations for the time-evolution of Green's functions. With this Hamiltonian, it is especially relevant to introduce the \textit{interaction picture}, in which both the operators and the wavefunctions have a time dependence. For the field operators we have :
\begin{equation}
	\hpsi(1) \equiv \hpsi(x_1, t_1) = e^{i\hat{H} t_1} \hpsi(x_1) e^{-i\hat{H} t_1}
\end{equation}
and for the interaction Hamiltonian :
\begin{equation}
	\hat{H}_I'(t) = e^{i\hat{H} t} \hat{H}'(t) e^{-i\hat{H} t} = \int dx dx' \hpsidag(x,t^+) U(x,x';t) \hpsi(x',t)
\end{equation}
where we use the notation $t^+$ for $t+\delta (\delta \to 0^+)$. We define a time-evolution operator in terms of this interaction Hamiltonian :
\begin{equation}
	\hat{S} = \exp\biggl\{ -i \int_{-\infty}^{+\infty} dt \hat{H}_I'(t) \biggr\}
\end{equation}
We now write the definitions of single- and two-particle Green's functions that include the dependence in $\Phi$ :
\begin{equation}
	G(1,2) = -i \frac{\bra{N}\hat{T}\left[\hat{S}\hpsi(1)\hpsidag(2)\right]\ket{N}}{\bra{N}T[\hat{S}]\ket{N}} \label{eq:GF}
\end{equation}
and
\begin{equation}
	G_2(1,2:1',2') = (-i)^2  \frac{\bra{N}\hat{T}\left[\hat{S}\hpsi(1)\hpsi(2)\hpsidag(2')\hpsidag(1')\right]\ket{N}}{\bra{N}T[\hat{S}]\ket{N}}\label{eq:2GF}
\end{equation}
where $\ket{N}$ is the exact ground state of the N-electron system and T is the Wick's time-ordering operator. It ensures that the time variable increases from right to left. It gives $\hat{T}\left[ \hpsi(\rr_1 t_1) \hpsidag(\rr_2 t_2) \right] = \theta(t_1 - t_2) \hpsi(\rr_1 t_1) \hpsidag(\rr_2 t_2) - \theta(t_2-t_1)\hpsidag(\rr_2 t_2)\hpsi(\rr_1 t_1)$, where $\theta$ is the Heaviside function. The physical meaning of the one-body Green's function is the probability amplitude that an electron added in the system at time $t_2$ and position $\rr_2$ propagates to position $\rr_1$ and time $t_1$. In the time-ordered formalism that we are using, it is also the probability amplitude that a hole created at time $t_1$ and position $x_1$ propagates to $(x_2,t_2)$, depending on how the two time variables are ordered.
%

We will now sketch a derivation for the equations of motion for the single-particle Green's function. To do this, we will make explicit the time dependence of each term in \ref{eq:GF}.
we start with the time evolution operator in the interaction picture, whose time dependence is :
\begin{equation}
	\hat{S}(t_a,t_b) = \exp	\biggl\{ -i \int_{t_a}^{t_b} dt \hat{H}_I'(t) \biggr\}
\end{equation}
and its time derivatives are:
\begin{align}
\begin{split}
	\frac{\partial}{\partial t_a} T[\hat{S}(t_a,t_b)] &= -i \hat{H}'_I(t_a)T[\hat{S}(t_a,t_b)] \\
	\frac{\partial}{\partial t_b} T[\hat{S}(t_a,t_b)] &= i T[\hat{S}(t_a,t_b)]\hat{H}'_I(t_b)
\end{split}
\end{align}
For the field operators, the derivation of the equations of motions can be found in Appendix \ref{app:EOM}. They read :
\begin{align}
\begin{split}
	\frac{\partial}{\partial t_1} \hpsi(1) &= -i \left[ h(1) + \int d3 v(1,3) \hpsidag(3)\hpsi(3) \right] \hpsi(1) \\
	\frac{\partial}{\partial t_2} \hpsidag(2) &= i \left[ h(2)\hpsidag(2) + \hpsidag(2) \int d3 v(2,3) \hpsidag(3)\hpsi(3) \right]
\end{split}	
\end{align}
where we introduced $v(1,2) = v(\rr_1,\rr_1)\delta(t_1-t_2)$ and $h(1)=h(\rr_1)$. Knowing that the derivative of the Heaviside function is a Dirac delta, we can write the equations of motion for the single-particle Green's functions :
\begin{align}
\begin{split}
	&\left[ i\frac{\partial}{\partial t_1} - h(1) \right] G_1(1,2) - \int d3 \Phi(1,3)G_1(3,2) + i \int d3 v(1,3) G_2(1,3^+;2,3^{++}) = \delta(1,2) \\
	&\left[ -i\frac{\partial}{\partial t_2} - h(2) \right] G_1(1,2) - \int d3 G_1(1,3)\Phi(3,2) + i \int d3 v(2,3) G_2(1,3^{--};2,3^-) = \delta(1,2)
\end{split}
\end{align}
with the notation : 
\begin{equation}
	\Phi(1,2) = \Phi(x_1,x_2;t_1) \delta(t_1-t_2).
\end{equation}
Here we notice that the equations for the single-particle Green's function depend on the two-particle Green's function. The latter could be expressed in terms of the three-body Green's function, and so on. Instead of using a hierarchy of higher-order Green's function, we will eliminate the two-particle Green's function from the equation and write a set of coupled integro-differential equations containing the self-energy and other useful quantities.
%

We use the functional derivative identity which is derived in appendix B \textcolor{red}{I don't know if it is necessary to include this derivation}:
\begin{equation}
	G_2(1,3;2,3^+) = G(1,2)G(3,3^+) - \frac{\delta G(1,2)}{\delta \Phi(3)}
\end{equation}
where we consider the external potential to be local in space $\Phi(x_1,x_2;t_1) = \Phi(x_1,t_1)\delta(x_1,x_2)$. This restriction is enough to generate the equations of motion for the single-particle Green's function. For the two-particle ones instead, one needs to consider the more general form of the external potential, non-local in space.
The equations of motion become :
\begin{equation}
	\left[ i \frac{\partial}{\partial t_1} - h(1) - \Phi(1) + i \int d3 v(1,3)G(3,3^+) \right] G(1,2) - i \int d3 v(1^+,3) \frac{\delta G(1,2)}{\delta \Phi(3)} = \delta(1,2)
\end{equation}
and
\begin{equation}
	\left[ i \frac{\partial}{\partial t_2} - h(2) - \Phi(2) + i \int d3 v(2,3)G(3^-,3) \right] G(1,2) - i \int d3 v(2^-,3) \frac{\delta G(1,2)}{\delta \Phi(3)} = \delta(1,2)
\end{equation}
We cannot take the limit $\Phi \to 0$ yet because it would require the knowledge of the functional dependence of $G$ on $\Phi$. However we can rewrite the equations of motion (or at least one of them and the other would undergo the same process) by introducing new physical quantities, coupled in nonlinear self-consistent equations. 
%

The first of these quantities is the \textbf{total potential} $V$ :
\begin{equation}
	\vtot (1) \equiv \int d2 v(12) \langle \hat{n}(2) \rangle + \Phi(1) \label{eq:vtot}
\end{equation}
where $\hat{n}$ is the density operator. The equation of motion for the Green's function is then :
\begin{equation}
	\left[ i \frac{\partial}{\partial t_1} + \frac{1}{2}\nabla^2(1) - \vtot (1) -i\int d3 v(1^+,3) \frac{\delta}{\delta \Phi(3)} \right] G(1,2) = \delta(1,2) \label{eq:GF_EOM}
\end{equation}
To get rid of the functional derivative with respect to the external perturbation, we make use of the definition of the inverse Green's function and of the functional differentiation of a product:
\begin{equation}
	\frac{\delta G(1,2)}{\delta \Phi(3)} = - \int d45 G(1,4) \frac{\delta G^{-1}(4,5)}{\delta \Phi(3)} G(5,2)
\end{equation}
where the inverse single-particle Green's function is defined as :
\begin{equation}
	\int d3 G^{-1}(1,3) G(3,2) = \int d3 G(1,3)G^{-1}(3,2) = \delta(1,2) \label{eq:inv_GF}
\end{equation}
We now use the chain rule for functional differentiation :
\begin{equation}
	\frac{\delta G^{-1}(4,5)}{\delta \Phi(3)} = \int d6 \frac{\delta G^{-1}(4,5)}{\delta \vtot(6)} \frac{\delta \vtot(6)}{\delta \Phi(3)}
\end{equation}
We introduce the \textbf{scalar vertex function}, a three-point quantity defined as :
\begin{equation}
	\Gamma(1,2;3) \equiv -\frac{\delta G^{-1}(1,2)}{\delta \vtot(3)} \label{eq:vertex}
\end{equation}
We introduce the \textbf{inverse dielectric matrix} $\inveps$ :
\begin{equation}
	\inveps(1,2) = \frac{\delta \vtot(1)}{\delta \Phi(2)}.
\end{equation} 
It is the many-body formulation of the classical (inverse) dielectric matrix. 
We introduce the \textbf{dynamically screened interaction} $W$ or screened Coulomb interaction, defined as :
\begin{equation}
	W(1,2) \equiv \int d3 \inveps(1,3) v(3,2) \equiv \int d3 v(1,3) \inveps(2,3)
\end{equation}
Note that the screened interaction is symmetric under the exchange of indices $W(1,2) = W(2,1)$.
Finally, we introduce the \textbf{electron self-energy}, defined as :
\begin{equation}
	\Sigma(1,2) = i \int d34 G(1,3) \Gamma(3,2;4) W(4,1^+)
\end{equation}
%

With these quantities, we can rewrite the equation of motion for the single-particle Green's function :
\begin{equation}
	\left[ i \frac{\partial}{\partial t_1} + \frac{1}{2}\nabla^2(1) - \vtot (1)\right] G(1,2) - \int d3 \Sigma(1,3) G(3,2) = \delta(1,2) \label{eq:GF_EOM_SE}
\end{equation}
We can see here that the self-energy $\Sigma$ has the meaning of a nonlocal and energy-dependent effective single-particle potential.

Using eqs. \eqref{eq:vertex} and \eqref{eq:GF_EOM_SE}, we can express the vertex function in terms of the above quantities. 
\begin{equation}
	\Gamma(1,2;3) = \delta(1,2) \delta(1,3) + \int d4567 \frac{\delta \Sigma(12)}{\delta G(4,5)} G(4,6) G(7,5) \Gamma(6,7;3).
\end{equation}
More details about these quantities and their derivations can be found for example in Strinati's review\cite{strinati1988application}.
The previous quantities form a set of coupled integro-differential equations. In order to close the loop and build a self-consistent set, we need to write the relations between $W$ and the other quantities. By combining $\vtot$, $\inveps$ and $W$, we get :
\begin{equation}
	W(1,2) = v(1,2) + \int d34 v(1,3) \frac{\delta \langle \hat{n}(3)\rangle}{\delta \vtot (4)} W(4,2)
\end{equation}
We define the \textbf{irreducible polarizability} to be :
\begin{equation}
	P(1,2) \equiv \frac{\delta \langle \hat{n}(1)\rangle}{\delta \vtot (2)}.
\end{equation}
The reducible polarizability would be the derivative of the density with respect to the perturbation $\Phi$. With the following relation 
\begin{equation}
	n(1) = \langle \hat{n}(1)\rangle = -iG(1,1+),
\end{equation}
and using properties of the inverse Green's function and the chain rule, one can write :
\begin{align}
\begin{split}
	P(1,2) &= -i \frac{\delta G(1,1+)}{\delta V(2)} = i \int d34 G(1,3)\frac{\delta G^{-1}(3,4)}{\delta V(2)}G(4,1^+)\\
	&= -i \int d34 G(1,3) G(4,1^+) \Gamma(3,4;2)
\end{split}
\end{align}
Then we can write the screened interaction in term of $P$ :
\begin{equation}
	W(1,2) = v(1,2) + \int d34v(1,3)P(3,4) W(4,2)
\end{equation}
as well as the dielectric matrix :
\begin{equation}
	\epsilon(1,2) = \delta(1,2) - \int d3 v(1,3) P(3,2)
\end{equation}
We now have a set of coupled self-consistent equations, where the limit $\Phi \to 0$ can be taken. 


\subsection{Dyson equation}

In order to be able to compute the Green's function and the related useful quantities, we need to reformulate the Green's function. We start by separating the part which comes only from the one-particle operators in the equation of motion \eqref{eq:GF_EOM_SE} :
\begin{equation}
	\left[ i\frac{\partial}{\partial t_1} - h(1) \right] G_0(1,2) = \delta(1,2)
\end{equation}
This is the definition of the non-interacting Green's function $G_0$. It also has an inverse and they obey the same relation as the full single-particle Green's function \eqref{eq:inv_GF}, therefore one can write :
\begin{equation}
	G(1,2) = \int d34 G_0(1,4)G^{-1}_0(4,3)G(3,2)
\end{equation}
Inserting the above equation in \eqref{eq:GF_EOM_SE}, we get :
\begin{equation}
	\int d3 \left[ G_0^{-1}(1,3) - \Sigma(1,3) \right] G(3,2) = \delta(1,2).
\end{equation}
After multiplying the from the left by $\int d1 G_0(4,1)$ we obtain :
\begin{equation}
	G(1,2) = G_0(1,2) - \int d34 G_0(1,3) \Sigma(3,4) G(4,2) \label{eq:Dyson}
\end{equation}
or equivalently,
\begin{equation}
	G^{-1}(1,2) = G_0^{-1}(1,2) - \Sigma(1,2)
\end{equation}
Equation \eqref{eq:Dyson} is called the \textit{Dyson equation} for the single-particle Green's function. Knowing $G_0$, which is numerically simple to compute, and approximating the self-energy $\Sigma$, which we will discuss later, allows one to compute the Green's function $G$.

\textcolor{red}{Should I put the GW approx subsection here and the QP equations after ? yes}
\subsection{The $\mathbf{GW}$ approximation}
With the Dyson equation for the Green's function, we can now complete the set of self-consistent, coupled equations that are called the Hedin's equations:
\begin{empheq}[box=\widefbox]{align*}
	&\Sigma(1,2) = i \int d34 G(1,3) \Gamma(3,2;4) W(4,1^+) \\
	&G(1,2) = \int d34 G_0(1,4)G^{-1}_0(4,3)G(3,2) \\
	&\Gamma(1,2;3) = \delta(1,2) \delta(1,3) + \int d4567 \frac{\delta \Sigma(12)}{\delta G(4,5)} G(4,6) G(7,5) \Gamma(6,7;3) \\
	&P(1,2) = -i \int d34 G(1,3) G(4,1^+) \Gamma(3,4;2) \\
	&W(1,2) = v(1,2) + \int d34v(1,3)P(3,4) W(4,2)
\end{empheq}
self-consistent cycle; consider $\frac{\delta \Sigma(12)}{\delta G(4,5)} = 0$ so that $\Gamma = \delta(1,2)\delta(1,3)$ and $\gS = iGW$. This is the so-called $GW$ approximation. Still self-consistent but we do only a one-shot. Starting point are the DFT KS eigenvalues. Linearize $\gS$ around the KS eigenvalues. Real part of SE is a shift to the KS eigenvalues. 


\subsection{Quasiparticle equations}
Once the limit $\Phi \to 0$ is taken, there is no time-dependent potential acting on the system of N electrons. Hence, the system is invariant under time translation and the Green's function depends only on the time difference $\tau = t_1 - t_2$. One can do the Fourier transform from time $\tau$ to frequency $\omega$, and we can write the Green's function in the so-called Lehmann representation :
\begin{equation}
	G(x_1,x_2;\omega) = \sum_a \frac{f_a(x_1)f_a^*(x_2)}{\omega-\varepsilon_a + i\eta} + \sum_i \frac{f_i(x_1)f_i^*(x_2)}{\omega-\varepsilon_i + i\eta}
\end{equation}
where $a,i$ denote electron states, $f_a(x) = \bra{N}\hpsi(x)\ket{N+1,a}$ and $f_i(x) = \bra{N-1,i} \hpsi(x) \ket{N}$ are the Lehmann amplitudes (as called Dyson orbitals). They are defined with the $N$-electron ground state $\ket{N}$ whose total energy is $E_N$, the electron state number $a$ of the $(N+1)$-electron system $\ket{N+1,a}$ with total energy $E_{N+1,a}$ and the electron-state number $i$ of the $(N-1)$-electron system $\ket{N-1,i}$ with total energy $E_{N-1,i}$.
The Lehmann representation of the Green's function also contains the quasiparticle energies, which are defined as $\varepsilon_a = E_{N+1,a} - E_N = -A_a$ and $\varepsilon_i = E_N - E_{N-1,i} = -I_i$. $A_a$ and $I_i$ are the electron affinities and ionization energies. This highlights the link between the poles of the Green's function from many-body perturbation theory and the photoemission and inverse photoemission spectroscopies. 

With this form of the Green's function we can reformulate the Dyson equation \eqref{eq:Dyson}. Just as the Green's function, the self-energy depends only on the time difference $\tau = t_1 - t_2$. We can then take the Fourier transform of eq. \eqref{eq:GF_EOM_SE} :
\begin{equation}
	\left[ \omega - h(\rr_1) \right] G(x_1,x_2;\omega) - \int dx_3 \Sigma(x_1,x_2;\omega) G(x_3,x_2;\omega) = \delta(x_1,x_2)
\end{equation}
Inserting the Lehmann representation of $G(x_1,x_2;\omega)$, one can select the term corresponding to a given pole $\varepsilon_k$ by multiplying the equation by $\omega - \varepsilon_k$ and taking the limit $\omega \to \epsilon_k$, giving :
\begin{equation}
	\left[ \epsilon_k - h(\rr_1) \right] f_k(x_1)f^*_k(x_2) - \int dx_3 \Sigma(x_1,x_3;\varepsilon_k) f_k(x_3) = \varepsilon_k f_k(x_1)
\end{equation}
and we obtain an eigenvalue equation :
\begin{equation}
	h(\rr_1)f_k(x_1) + \int dx_3 \Sigma(x_1,x_3;\varepsilon_k)f_k(x_3) = \varepsilon_k f_k(x_1)
\end{equation}
\textcolor{red}{Maybe frame this equation} 
These are the \textit{quasiparticle equations}. They give the quasiparticle energies and the Lehmann amplitudes, or also called the Dyson orbitals. Here it is made apparent that $\Sigma$ is a nonlocal, energy-dependent potential. One thing to notice is also the fact that the quasiparticle energy is made out of the bare, independent single particle, and another term coming from the interaction with surrounding particle. The quasiparticle is the bare particle "dressed" with the interaction.
\textcolor{red}{Linearize the SE, the Z factor appears. talk about the fact that the KS eigenvalues are the starting point}
\textcolor{pink}{Imaginary part of self-energy is lifetime. Cite Mattuck ?}
\textcolor{pink}{Subtract the $v_{xc}$ from DFT when computing QP energies}

\textcolor{red}{Relation between the 2-particle GF and the linear response function}



