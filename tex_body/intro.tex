\chapter*{Introduction}
\addcontentsline{toc}{chapter}{Introduction}
Modern society is facing huge challenges. Climate change and the need for cleaner energy to reduc green-house gas emissions, clashing with the hope of permanent economic growth and global exchange of information, goods and people. Unprecedented issues arise from this more and more advanced and interconnected civilization. This PhD work was initiated in the middle of the most impactful sanitary crisis of the contemporary era, the COVID-19 pandemic. 
Deep ultraviolet is a radiation that kills the coronavirus. (Show market share of UV emitters and applications)
Moreover, technological innovation could bring a way to produce cleaner energy, such as fusion which is a priority research topic, or photovoltaic, but also energy storage, preferably without the need to extract rare materials. Effort should be put into the use of sustainable and abundant base material.
Importance of theoretical estimation of the capabilities of certain materials for optoelectronic devices fabrication (check intro of Kioupakis talk at EPFL)
hBN is promising for its properties and is not rare
%Generalities about hBN \\
We could compute phonon-assisted absorption but the phonon satellites are very small wrt to the main peaks so when they overlap, they are not visible.\\
Theory of phonon-assisted optics : it exists. For instance, Williams-Lax theory that treats lattice-dependent band features and phonon-assisted transitions 
There is also Hallen-Bardeen-Blatt and Heine-Cardona but I don't know too much about them.
Besides, some models of exciton-phonon coupling exists also, some specifically for luminescence, but they are computationally expensive for real materials.
==> there is a need for exciton-phonon coupling and phonon-assisted optics from first principles. In chapters 2 and 3, two different approaches are presented in section \textcolor{red}{ref to sections}. The first one consists in calculating the response function correction due to exciton-phonon coupling from finite differences, by displacing atoms in supercells. The second one is a more general approach based on \acrshort{MBPT}, in which a dynamical correction induced by electron-phonon coupling is added as a perturbation to the Bethe-Salpeter kernel. Thanks to this \textit{ab initio} framework, the scattering of every exciton with every phonon mode can be computed, over the whole Brillouin Zone. These calculations are done in the unit cell.
Both approaches allow to compute the exciton-phonon coupling. Then, the absorption can be obtained after post-processing of a standard \acrshort{BSE} calculation. To obtain the luminescence spectra, we use the van Roosbroeck -- Shockley relation for both cases.
Exciton-phonon in hBN are not only possible, but likely due to strong electron-phonon coupling. Verified experimentally by the External Quantum Efficiency comparable to direct gap materials.\\

%
electronic and optical properties of layered materials (an in particular BN) can be modified by many factors
strain (cite something), stacking (cite sponza2018direct), twisted angle (cite latil2023structural, impellizzeri2022electronic), substrate interaction etc...\\
%
hBN : hig gap, pDOS of valence at K is mostly Nitrogen, $\pi$ orbital(overlap of $p_z$, bonding)  . pDOS of conduction at M is mostly Boron, $\pi^*$ orbital (overlap of $p_z$ with opposite coefficient, antibonding)
%
%
% ETSF webinar of Fulvio
% indirect bandgap only; we can write the response function with derivatives of the response function because we consider only phonon-assisted transitions; need supercells
% PLE (propto absorption) or absorption and PL are not symmetric, as expected for direct gaps
% the minimum exciton at T is the degenerate dark state at Gamma which splits
% for absorption, the bright exciton overlaps with the phonon replicas and it is so bright that we don't see them 

In condensed matter, the problem of exciton-phonon coupling and phonon-assisted luminescence is an old topic. The first studies date back to the $60$s by Toyozawa \emph{et al.}\cite{toyozawa2003optical,toyozawa1964interband} and the first dynamical solution of the Bethe-Salpeter equation (BSE), the so-called Shindo solution, was proposed precisely to study the exciton-phonon problem.\cite{shindo1970effective}
More recently, this problem has been studied in different materials, from nanotubes\cite{perebeinos} to 2D crystals, and new methodologies were introduced, such as the cumulant Ansatz\cite{cudazzo2020first}, polaron transformation,\cite{feldtmann2009phonon} density matrix,\cite{brem2020phonon} two-particles Green's functions\cite{antonius2017theory} and real-time approach,\cite{paleari2022coupling} with the aim of deriving a modern formulation of exciton-phonon interaction and its related observables, such as phonon-assisted luminescence, within many-body perturbation theory (MBPT).