\chapter*{Introduction}
\addcontentsline{toc}{chapter}{Introduction}
Generalities about hBN \\
We could compute phonon-assisted absorption but the phonon satellites are very small wrt to the main peaks so when they overlap, they are not visible.\\
Theory of phonon-assisted optics : it exists. For instance, Williams-Lax theory that treats lattice-dependent band features and phonon-assisted transitions 
There is also Hallen-Bardeen-Blatt and Heine-Cardona but I don't know too much about them.
Besides, some models of exciton-phonon coupling exists also, some specifically for luminescence, but they are computationally expensive for real materials.
==> there is a need for exciton-phonon coupling and phonon-assisted optics from first principles. In chapters 2 and 3, two different approaches are presented in section \textcolor{red}{ref to sections}. The first one consists in calculating the response function correction due to exciton-phonon coupling from finite differences, by displacing atoms in supercells. The second one is a more general approach based on \acrshort{MBPT}, in which a dynamical correction induced by electron-phonon coupling is added as a perturbation to the Bethe-Salpeter kernel. Thanks to this \textit{ab initio} framework, the scattering of every exciton with every phonon mode can be computed, over the whole Brillouin Zone. These calculations are done in the unit cell.
Both approaches allow to compute the exciton-phonon coupling. Then, the absorption can be obtained after post-processing of a standard \acrshort{BSE} calculation. To obtain the luminescence spectra, we use the van Roosbroeck -- Shockley relation for both cases.
Exciton-phonon in hBN are not only possible, but likely due to strong electron-phonon coupling. Verified experimentally by the External Quantum Efficiency comparable to direct gap materials.